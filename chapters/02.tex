\chapter{Attributive constructions: Typological and Semitic perspectives}
\label{ch:attributive}
\largerpage
\section{Theoretical framework} \label{ss:theoretical_framework}

The current research, while informed by advances in \isi{linguistic typology}, is situated methodologically within the structuralist current of linguistics, which analyses language as a system of oppositions, be they between the consecutive constituents of  an utterance (the so-called \concept{syntagmatic axis}) or between possible constituents at a given point in discourse (the \concept{paradigmatic axis}). The usage of these analytical concepts will be further clarified below. While the structuralist tradition is interested mainly in the description of a single language, one can profitably extend these tools to compare and contrast several languages, as I do in the current study.

In structuralist linguistics, \concept{morphemes} are defined by way of opposition along the paradigmatic axis: Whenever an element of language (a \concept{signifiant} in Saussurian terms) stands in opposition in a given \concept{environment} (or \concept{syntagm}) with other elements and its exchange by these other elements co-varies systematically with a difference in \textit{meaning} broadly conceived (a \concept{signifié}), this element can be identified as a \concept{linguistic sign}. Minimal signs, i.e.\ those that are not analysable in terms of smaller signs, are considered to be \concept{morphemes}. As simple as this procedure is, it is possible to apply it in quite different ways, in accordance with the understanding of the term \concept{environment} used above. If by \concept{environment} one means a well-formed utterance, one gets the classical notion of a \concept{structural paradigm}. If, on the contrary, one allows for opposition within smaller environments, such as word-forms, the above procedure yields the notion of a \concept{morphological paradigm}. The elements identified as morphemes would be different in each case: For instance, a grammatical \isi{case marker} attached to a noun-stem, whose usage is obligatory in certain environments may not be considered a morpheme within the classical approach (as it does not stand in opposition within a well-formed utterance), but it can be seen as a morpheme in the morphological approach, as the word in isolation shows variation in case. In this study I opt for the latter approach: namely, morphemes are defined relative to word-forms in isolation, and not relative to full utterances.  

Linguistic structuralism is normally equated with no \textit{a priori} categories of language, as these should be defined on a per-language basis using the analytical method sketched above. Yet,  as this research follows the footsteps of previous scholars, I will operate within a framework assuming the existence of three general grammatical relations in language, described in the next section. Of these, the \concept{attributive relationship} shall be seen as the abstract functional correlate of the concrete grammatical patterns examined, the \concept{attributive construction}\textsc{s} (ACs). This is further explained in \ref{ss:attr_relation}. In structural terms, the \isi{attributive relationship} is the \concept{signifié} 
of several different \concept{signifiant}\textsc{s}. 

The rest of the chapter is devoted to anchoring these terms in the traditions of \isi{linguistic typology} (\ref*{ss:AC_typ}) and of \ili{Semitic} linguistics  (\ref*{ss:AC_sem}). \Ref{ss:typology_here} synthesizes from these different approaches one methodology used in this study. 

\subsection{The three relations} \label{ss:threeRel}

In this book, I rely on a simple dependency model of morpho-syntax admitting three basic dependency relations holding between elements of a clause: 

\begin{enumerate}
\item The \concept{predicative relation}, holding between a subject and a predicate.

\item The \concept{attributive relation}, holding between a head and its attribute.

\item The \concept{completive relation}, holding between a predicative construction and a complement.
\end{enumerate}

This theory, as presented here, was advocated by the Israeli linguist Gideon Goldenberg (1930--2013), who himself credited the German philologist Karl Ferdinand Becker (1775–1849) as being one of its early forefathers \parencites(see)()[10]{BeckerGrammar}{GoldenbergRelations}[Ch.\ 11]{GoldenbergSemitic}[37--38]{CohenSha}{GutmanReview}. 

Goldenberg saw the theory as both general in scope and at the same time especially adequate to the \ili{Semitic} family:\footnote{Goldenberg clearly saw these relations as valid cross-linguistic notions, but he  did not address the question whether they represent linguistic universals, nor did he tie them to any nativist conception of language. It seems, rather, that  the cross-linguistic validity of these notions stems from the fact that they represent syntactic correlates of necessary communicative functions of language such as assertion (the \isi{predicative relation}), qualification of referents (the \isi{attributive relation}) and of events (the \isi{completive relation}).} 

 \blockquote[{\citealt[142]{GoldenbergSemitic}}][.]{The recognition of three essential types of grammatical relations, or bonds, has been a major approach to syntactic analysis commonly pursued in linguistics during the last two centuries. With regard to \ili{Semitic} languages and in connexion with case declension such a conception appositely reflects the very structure of the languages involved} This is so, since the case-marking classical \ili{Semitic} languages (\CArab, \ili{Akkadian} and \ili{Ugaritic}) have exactly three cases, which correspond well with the three mentioned relations. Regarding linguistic change in \ili{Semitic} languages, he asserted that by using this theory  \blockquote[{\citealt[142]{GoldenbergSemitic}}][.]{we may be able to better understand the meaning of changes in some innovative languages and thus perhaps even to measure typological innovation}. Since the aim of this book is exactly to investigate change in modern \ili{Semitic} languages, namely the \ili{NENA} branch, the usage of this theory seems especially adequate. 

To the three above-mentioned dependency relations, one must add \concept{apposition}, not being a dependency relation, but rather an equivalence relation. In this framework, two elements are considered to be in \concept{apposition}, whenever both are governed by the same dependency relation, share potentially the same grammatical features (number, gender, case and \isi{definiteness} -- if explicitly marked), and are co-referential. In such conditions, they can replace each other syntactically, although the two may not be equivalent on semantic and discursive grounds, as one element may take a higher information load. In the latter case, my notion of \isi{apposition} is similar to the notion of \enquote{appositional modification} defined by \citet[13]{Riessler} as following: \enquote{Semantically, the appositional modifier is headed by the modified noun.
Syntactically, however, the appositional modifier has an empty head which is
co-referential with the \isi{head noun} of the apposed noun phrase.} His \enquote{empty head} is analysed in the current framework as a covert pronominal element.\footnote{See also Cohen's definition of \isi{apposition} given in \vref{ft:Cohen_Apposition}. \citet[65]{Acuna-Farina1999} rejects a similar notion of \isi{apposition} claiming that \enquote{[i]f a syntactic relationship of this type is to be posited, then that relationship must be applicable to a number of other constructions, and
not just to one construction}, and contrasts it with the widely applicable notion of dependency. Yet, in the current framework, \isi{apposition} is not a syntactic relation \textit{sensu stricto}, i.e.\ a dependency relation, but rather an equivalence relation. Compare again with \citet[38]{CohenSha}: \enquote{It must be stressed that \isi{apposition} [...] is not in itself a relationship, but rather a repetition of a syntagm, and occasionally, of the relationship itself.} As such, the notion of \isi{apposition} is applicable to a wide range of constructions.}

\subsection{The attributive relationship and its manifestation}
\label{ss:attr_relation}

In this book, I am interested only in one of the relations mentioned above, namely the \concept{attributive relation}. This is the dependency relationship within the NP domain which holds between a \isi{head noun} (or pronoun) and a second nominal element (the \concept{attribute} or \concept{dependent}) qualifying the \isi{head noun} \parencites(cf.)()[1--2]{GoldenbergAttribution}{CohenAttribute}. This notion is closely related to Jespersen's notion of \concept{junction} \citep[Ch.\ 8]{JespersenPhilosophy}. Note that semantically, there is no restriction on the type of the qualification involved, which may range from possession to qualification of some property of the head. 



In structuralist terms, the \isi{attributive relation} is a \concept{signifié} (function) whose \concept{signifiant} (form, i.e.\ morpho-syntactic exponent), is  an \concept{attributive construction} (=AC).
 I use the term \concept{construction} here as denoting a linear ordering of segmental (and possibly super-segmental) material together with paradigmatic slots, \enquote{place-holders} so to speak, which can accommodate either an open group of elements (i.e., a lexical paradigm, often corresponding to some part of speech), or a closed group of elements (i.e., a functional paradigm, often corresponding to an inflectional paradigm of some morpheme), to which a specific function is tied.\footnote{The notion of \concept{construction} was popularized by the proponents of Construction Grammar \citep[e.g.][]{GoldbergConstructions,CroftRadical}. It corresponds in fact to an abstract understanding of the long-standing Saussurian notion of \concept{linguistic sign}, i.e. a coupling of \concept{form} and \concept{function}.} In a given language, one may find many different attributive constructions. While all of these encode an \isi{attributive relationship}, they may be  used in different syntactic contexts, or convey different semantic or stylistic nuances. I will only linger upon these differences as far as they are insightful for my comparative purposes. 

Every AC is defined as having two paradigmatic slots corresponding to the head and the attribute. In some cases, however, the two elements in question are split across two separate NPs which stand in \isi{apposition} to each other, rather than in an \isi{attributive relationship}.   In such a configuration, it is often the case that the \isi{attributive relationship} holds within one of the NPs, in which the other NP is being represented pronominally. In virtue of this, it is possible to identify one NP as being the qualified, and the second NP as being the qualification,\footnote{It is often the case that the qualifying NP follows in discourse the qualified NP, in accordance with the general tendency of pronouns to be anaphoric rather than cataphoric.} and posit an \concept{indirect attributive relationship} between them (cf.\ the term \concept{indirect annexation} used by \cite[79]{GoldenbergEarly}).
Yet  in such a case one cannot accurately use the terms \emph{head} and \emph{attribute}\is{attributive} for these NPs, as they imply a direct \isi{attributive relationship} between the two elements. To overcome this terminological problem I shall use the notions of \concept{primary} and \concept{secondary} to denote the two members, in line with \citet[38]{PlankIntro}:\footnote{These terms clearly bear affinity to the same terms introduced by \citeauthor{JespersenPhilosophy}, but note that his usage is broader, as it applies equally well to cases of junction as well as nexus \citep[97]{JespersenPhilosophy}.}

\blockquote{The nominals in relation will be neutrally referred to as \concept{primary} and \concept{secondary}. Attributes are prototypical secondaries
vis-a-vis their heads [...] but on referential and distributional grounds, secondary rank is also justified for the appositum in \isi{apposition} or for a nominal indirectly related to
another as a secondary predicate or the like.}

Prototypically, the \prim and the \secn are expected to be nouns, but this does not exclude other nominal elements, chiefly pronouns and adjectives. Moreover, as we shall see, the \secn can also be  a \isi{prepositional phrase} (a PP), or even a clause (Cl). Moreover, in \ili{Semitic} languages the same constructions are often used with \isi{adverbial} elements (prepositions or conjunctions) as \prims. While these uses can be seen as peripheral, and not strictly realizing an \isi{attributive relationship}, they are sometimes illuminating for the study's comparative purposes, and thus will be taken into account.



\section{Attributive constructions from a typological perspective} \label{ss:AC_typ}
\largerpage
The notion of the \isi{attributive relationship}, and the corresponding ACs,  is clearly a very broad, unifying concept. Many typological studies, on the other hand, look at a restricted set of ACs as their object of study. Thus, \citet{UltanPossession} establishes a typology of \concept{possessive constructions}, i.e.\ ACs whose \secn is a nominal possessor.\footnote{Ultan elaborates a quite complex typology, taking into account both the \concept{locus} of marking (see \sref{ss:head_vs_dep}), and the type of marking: whether it is syntactic or morphological on the one hand, and whether it indexes features of the possessed noun or the possessor. Yet  at the end he reduces this typology to a simple \concept{locus} typology. Unfortunately, the lack of clear definitions of the various marking categories and the sparse use of examples renders his work less than insightful.}  More recently, \citet{Tamm2003Possessive} discusses the \textit{Possessive noun phrases in the languages of Europe}. A similar restriction is taken by \citet{NicholsBickelWals}, discussed further below. The restriction applied is basically a semantic one. Another quite common division of the domain of ACs is according to the syntactic category of the \secn: Dryer, looking at word-order phenomena, separates ACs whose \secn is a noun (a \enquote{genitive}), an adjective or a \isi{relative clause} \citep{DryerWALS86, DryerWALS87, DryerWALS90}. \citet{GilWals60}, on the other hand, examines to what extent these three categories are differentiated across the languages of the world.\footnote{\citet[235]{GoldenbergSemitic} comments on Gil's approach: \enquote{[t]he constitutional identity of constructions with genitive nominals, adjectives, and relative complexes will in any case belong to the profoundest level of language structure, not to be regarded as different semantic types of attributions that collapsed due to imperfect differentiation}. }



\newpage 
In the framework of Canonical Typology, \citet{NikolaevaSpencer} pose inalienable possession and attributive modification (by adjectives) as two separate canonical constructions,\footnote{Note that their use of the term \concept{construction} is different than ours, as it is not tied to a specific manifestation in language.} while alienable possession and modification by noun are their non-canonical counterparts. While they too split the AC domain, using both  semantic and syntactic criteria, they do acknowledge that \textquote[{\cite[209]{NikolaevaSpencer}}]{there is some deeper link between the two constructions}, by examining languages in which these functions are expressed identically, specifically the {Ezafe} marking of \ili{Iranic} languages (which shall be examined carefully in this book; see \sref{ss:ezafe_dispute}).

It is not surprising that large scale typological studies attempt to focus on a restricted domain of constructions, using various semantic and syntactic criteria to delimit it. Such criteria, in line with Haspelmath's notion of \concept{comparative concept}\textsc{s} \citep{HaspelmathComparative}, assure the typologists they are comparing like with like. As this study is focused on a restricted and similarly shaped set of languages (namely, the \ili{NENA} languages and their contact languages), I have had the leisure of defining a broader object of study. Of course, more comprehensive accounts can be found in the typological literature as well. Thus, \citet{Fairbanks} starts by treating equally cases of adnominal modification by nouns, PPs and clauses, although his main interest is nominal modification. Another broad account,  discussed in more detail below, is that of \citet{PlankIntro}.

In the following, I shall examine in more detail the typologies of \citet{NicholsBickelWals} and \citet{PlankIntro}. To round of the picture, I shall present also the recent typology elaborated by \citet{Riessler}.

\subsection{Head-marking vs.\ dependent-marking typology} \label{ss:head_vs_dep}
\largerpage[2]
Following the pioneering work of \citet{NicholsHead},  \citet{NicholsBickelWals} classify possessive constructions (and subsequently languages\footnote{For each language, they consider only one construction, which \enquote{is [the] default or has the fewest restrictions} \citep[\S 2]{NicholsBickelWals}.}) according to the \concept{locus} of marking, i.e.\ whether the construction is marked morpho-syntactically on the head (\prim in the current terminology) or the dependent (\secn), irrespective of the order of these two constituents.\footnote{Such a classification has in fact been proposed earlier by \citet{UltanPossession}, but in less clear terms.} Since the marking of each \isi{locus} is independent, this simple typology yields 4 types of marking: \isi{head marking}, \isi{dependent marking}, \isi{double marking} and no marking.\footnote{A fifth category, is dedicated to \enquote{low-frequency but systematic further patterns} \citep[\S 1.5]{NicholsBickelWals}. These are cases where the markers could not be easily associated with either the head or the dependent, but they represent only 2.5\% of their sample (i.e.\ 6 languages).} 

\newpage 
While this typology succeeds  at capturing large geographical distributions, it suffers from two shortcomings rendering it somewhat simplistic at the descriptive level. 

First, there is no differentiation between syntactic marking and morphological  marking. Rather, the authors agglomerate the two under the heading \enquote{overt morphosyntactic marking}.\footnote{By syntactic marking I mean marking achieved by a separate syntactic element, typically bearing phrasal scope, while morphological marking is achieved by inflection of a word, possibly (but not necessarily) having narrow scope on that word only. Recently, \citet{HaspelmathWord} has suggested that this distinction is void and cannot be applied consistently across various languages, yet I find it important in studies of \isi{language change}, like the current one.} 
Thus, the preposition \transc{of} in the \ili{English} phrase \transc{the price of oil} is considered to be a case of dependent-marking, probably due to its syntactic co-constituency with the dependent, on par with an inflectional genitive \isi{case marker}.\footnote{Cf.\  \citet[36]{Fairbanks}: \textquote{The main distinction between the genitive inflection and the pre/postposition is that the genitive inflection is inseparable from its noun and it must be repeated in certain situations.}} Since the syntactic constituency of an element may be disputed in some cases (especially if it cliticizes to another element), this can lead to analytical difficulties.\footnote{Indeed, as we shall see in \sref{ss:ezafe_dispute}, such a controversy exists around the \Per \ez* particle. \citet{NicholsBickelWals}  classify it without further comment as a head-marking instance.} 

Secondly, the typology does not differentiate between two quite distinct types of markers:  pure relational markers versus pronominal markers, which represent the antipodal \isi{locus} (i.e.\ the opposite member of the construction). Thus, in \ili{Turkish}, in which one finds both head marking and dependent marking, the two markers are of quite distinct type:


\acex[\Turk]
{Noun}{Noun}{Turk1}
{çocuğ-\opt{un} araba-sı}
{child-\opt\gen{} car-\poss.3 }
{\opt{the} child's car}
{Bozdemir}{49}

\largerpage
The dependent-marker is purely relational (a \isi{genitive case}),\footnote{The usage of the \isi{genitive case} in \ili{Turkish} is in fact not obligatory. When it is used it usually marks the \secn as definite and specific. In this it is similar to \ili{Turkish} \acc* case, which marks only definite objects. See \citet[49f.]{Bozdemir}.}  while the head-marker is pronominal. This is crucial, since the expression \transc{araba-sı} is by itself a well-formed NP meaning \transl{his car}. A similar criticism is made by \citet[229]{GoldenbergSemitic}:\blockquote{Attributive, or possessive, syntactic relations are commonly regarded as being marked either on the head or on the dependent attribute, not only by stem form or case, but also by personal morphemes, as if the possessive relation in \enquote{the man \textsuperscript{his-}house} (for \enquote{the man's house}) is marked by \enquote{his} on the head term \enquote{house} [\ldots] Pronominal morphemes, however, like other nominals or nominalizations, are not markers of the head-dependent relation, but belong with the \textit{termini} [=loci\is{locus}, \prim or \secn{}] between which the relation is apprehended.}

These two shortcomings were addressed to some extent by a more elaborate typology, presented in the next section.

\subsection{Plank's adnominal typology} \label{ss:Plank_typology}

A more elaborate typology of adnominal modification is presented by \citet{PlankIntro}. It is not restricted to a specific semantic domain of ACs, and indeed no specific restrictions are put on the \secn, except it being of nominal nature. As \citet[38]{PlankIntro} puts it: \blockquote{The following taxonomy of marking patterns is therefore
intended to be neutral (a) as to whether the nominal to be related to
another is a noun or something else (such as a derived adjective), and (b) as to
whether its relationship is one of [direct] attribution or of some other kind (such as
\isi{apposition}) --- and indeed, whether this relationship is that of an immediate
adnominal constituent or not.}

As mentioned above (\sref{ss:attr_relation}), Plank re-introduces the terms \textsc{primary} and \textsc{secondary} to refer to the two nominal members of the construction, which he symbolizes as X and Y, a practice I shall follow below.

Disregarding the word order of the two elements, Plank opts for an elaborate head vs.\ dependent marking typology, in which he differentiates between pure relational markers and pronominal markers, which he calls \concept{relatedness-indicators}.\footnote{I use here the term \emph{pronominal} in the basic meaning of representing (and possibly substituting) a noun.
In a similar typology proposed by \citet[ch.\ 2]{Riester}, such markers are termed \concept{agreement markers}. Riester's  typology is further elaborated in that it distinguishes between \concept{local agreement}, i.e.\ a marker exhibiting features of its own constituent, and \concept{non-local agreement}, i.e.\ a marker exhibiting features of the other constituent. Only the latter would be considered \concept{relatedness-indicators} in Plank's system.} 

\begin{modquote
The
relations identified may be those of secondary [pure \isi{dependent marking}] or of primary [pure \isi{head marking}]
 or of both [pure \isi{double marking}], with the markers normally associated, morphologically
or syntactically, with the respective nominals themselves. Relatedness-indicators
may occur on the secondary [pronominal dependent marker], reflecting some property of the
primary that it belongs with (such as its number, gender/class, person, or case);
or they may be on the primary [\isi{pronominal head} marking], reflecting some property of the secondary that it belongs with, or on both [pronominal \isi{double marking}]. {\citep[38]{PlankIntro}}
\end{modquote}

Since each one of the 4 markers type is in principle independent of the others, this yields 16 construction types. In fact, in the case of no marking at all, Plank distinguishes between syntactic \isi{juxtaposition} of the \prim and \secn (X\#Y), and morphological compounding of them (X+Y). However, except for this distinction, and judging from the citation above, syntactic and morphological markings are considered alike.\footnote{Thus \citet[656]{Tamm2003Possessive} commenting on Plank's work, writes: \enquote{[this] taxonomy does not distinguish morphological boundedness and syntactic association.}}
Plank acknowledges, however, the possibility of a third \isi{locus} of marking, namely \textquote[{\cite[39]{PlankIntro}}]{markers of the entire construction, linking
primary and secondary without forming a morphological co-constituent of either [...] (\enquote{associative} markers or lexical items such as `thing', `possession/belong', 'place')}. He dubs these items \enquote{links}. The elements may themselves carry pronominal markers of the \prim, the \secn or both, adding 4 more construction types.\footnote{He does not list construction types where both the link and the \prim or \secn are marked. If all these combinations would be marked, there would be 64 construction types. }

\textit{Prima facie}, one would expect such \enquote{links} to form a syntactic co-constituent with the \secn (see \ref{ss:Analytic_AC}), so it is not  clear what distinguishes them from normal secondary marking in Plank's typology.\footnote{Indeed, later on Plank acknowledges this difficulty: \textquote[{\cite[51]{PlankIntro}}]{No 7. [X Y-sec-x], [is] not always easily distinguished from No. 17, X Link-x Y}. \label{ft:lnk_difficulty}} Examining the accompanying examples does not clarify this point. Notwithstanding this possible confusion, Plank's typology is important in that it raises the question of distinguishing morphological marking vs.\ syntactic marking, and more importantly, it makes a clear difference between pronominal markers and pure relational markers. The typology adopted in this study is based to a large extent on Plank's typology. 

\largerpage[-1]
\subsection{Rießler's typology of attribution marking}


\begin{table} 
 
\begin{tabular}{l l}
\toprule
Juxtaposition & $X\ Y$ \\

Incorporation & $X+Y$ \\

Linker & $X\ \lnk\ Y$ \\

Anti-\isi{construct state} & $X\ Y_{\textsc{anti-cst}}$ \\

Anti-\isi{construct state} agreement & $X\ Y_{\textsc{anti-cst+agr}}$ \\ 

Construct state & $X_{\cst}\ Y$ \\

Double construct & $X_{\cst}\ Y_{\textsc{anti-cst}}$ \\

Head-driven agreement & $X\ Y_{\agr}$ \\

Possessor agreement & $X-y_{\poss}\  Y$ \\

\bottomrule
\end{tabular}

\caption{Rießler's typology of attributive marking}\label{tb:Riessler_typology}
\end{table}



As a last point of reference, I shall examine a recent typology of attribution marking elaborated by \citet[Ch.\ 4]{Riessler}. While Rießler focuses on \isi{adjectival attribution}, his typology covers larger ground, and is thus relevant for this study. In many respects, Rießler's typology is equivalent to Plank's typology discussed above, except that he uses a more technical, and to some extent obscure, terminology.\footnote{Judging by the lack of citation of \citet{PlankIntro} in Rießler's work, it seems his typology was elaborated independently of Plank's work.} I discuss it here, nonetheless, in order to compare the current framework (derived by and large from Plank's) to another recent typological framework of the same domain.

\citet[62]{Riessler} considers three main dimensions which characterize attributive markers, quoted hereby:

\begin{itemize} 
\item \textit{Syntactic source}, i.e., the central syntactic operation which constitutes attribution
and belongs either to \textit{agreement marking} or \textit{government}. [...]

\item \textit{Syntactic pattern}, i.e., devices projecting adjective phrases versus devices
projecting full noun phrases [...]

\item \textit{Syntactic \isi{locus}} of the respective formatives.
\end{itemize}


\largerpage[-1]
By \textit{syntactic source}, Rießler refers to the same distinction which Plank made between relational markers and (pronominal) relatedness markers. Pure relational markers are qualified by him as being issued by \enquote{government} of the entire construction ([+ \textsc{gov}] in his terminology), while relatedness markers are exponents of agreement ([+ \agr]). In cases where both marker types accumulate on one member, such as Plank's [X Y-sec-x] construction, he sees the agreement as being \enquote{secondary}.\footnote{In a quite unfortunate terminological decision, Rießler terms the pure relational markers \enquote{construct state} markers, whether they appear on the head or on the dependent (in which he calls it \enquote{anti-construct state}). As we shall see in the next section, the \isi{construct state} category is better reserved for use as a head-marking device, while the notion of \concept{case} should be used for dependent-marking. It seems that Rießler is reluctant to use the term \enquote{genitive case} to term the dependent-marking relational marker since he associates it with possessive semantics and his work focuses on \isi{adjectival attribution} \citep[see][43]{Riessler}.}

The second dimension, \textit{syntactic pattern}, is a novelty of this typology. Note, however, that it is specific to \isi{adjectival attribution}, and moreover, it assumes that there is a clear distinction between adjective phrases and (full) noun phrases. As we shall see  in \sref{ss:Goldenberg_typology} below such an assertion is not evident for the \ili{Semitic} languages which are the object of the current study.

The third dimension, \textit{syntactic \isi{locus}}, refers to the position of the marker: either on the head or the dependent. Just as Plank (and following \cite{NicholsHead}), \citet[59f.]{Riessler} recognizes a third \enquote{floating} locus of marking, not associated with any member of the construction. He terms such markers \enquote{linkers} and brings the Tagalog \transc{na/-ng} attributive marker as an example.\footnote{This marker appears in the second position of the NP, irrespectively whether the NP has the order attribute+head or head+attribute.} Interestingly, he notes that such true \enquote{linkers} are not found within his survey of the languages of northern Eurasia (see in this respect \vref{ft:lnk_difficulty}).

Rießler does not in general classify the markers according to their binding nature (syntactic or morphological), though interestingly, just as Plank he distinguishes between syntactic \isi{juxtaposition} (Plank's X\#Y) and morphological incorporation (Plank's X+Y) \citep[29--32]{Riessler}.

Using these different criteria, Rießler elaborates a typology of 11 different \isi{attributive construction} types (not counting \isi{double marking}). Disregarding those which are specific for \isi{adjectival attribution}, the remaining constructions are presented in \vref{tb:Riessler_typology}, alongside with an adaptation of Plank's notation for these constructions. This table can be compared to \vref{tb:main_AC_strategies} which presents the \isi{attributive construction} labels used in this study.

\section{Attributive constructions from a Semitic perspective} \label{ss:AC_sem}

In the \ili{Semitic} language family, the typical attributive construction is the \concept{annexation} construction, or \concept{construct state construction} (=CSC),
 in which the \prim  is marked by a special morphological form, the \concept{construct state}.\footnote{The use of the term \concept{construct state} alone to name the construction is thus misleading. Unfortunately, such a usage is prevalent in certain formal schools of linguistics, and even made it way to the Encyclopedia of \ili{Arabic} Language and Linguistics \citep{BenmamounConstruct}.} In \ili{Semitic} grammars, the \prim in this construction is normally called \concept{nomen regens}, and the \secn \concept{nomen rectum}, but I will stick to the terms \prim and \secn. 


\subsection{Relational nouns and the category of state} \label{ss:state}

In \ili{Semitic} languages, nouns (as well as other nominals) are inflected not only for the familiar categories of \textsc{number}, \textsc{gender} and possibly \concept{case}, but also for the category of \concept{state} \parencites(cf. for \ili{Hebrew})()[579]{HeckeConstruct}[581]{DoronConstruct}. In contrast to the other categories, the category of state is not \concept{projecting}, i.e.\ it is invisible for elements outside the NP in which it occurs. It may be for this reason that it has often been ignored in linguistic studies as a fundamental morpho-syntactic category of language, although it is in fact not restricted to \ili{Semitic} languages.\footnote{But see \textcites[344]{RetsoState}[268, especially fn.\ 3]{RetsoPlural}, who treats the notion of state as a morphological category in \ili{Semitic} languages, albeit as representing \enquote{allomorphic variation}, as he relies on the notion of \isi{structural paradigm}s (see \ref{ss:theoretical_framework}). The converse position, denying the validity of the category of state, is  advocated by \citet[318]{FaustAt}, who claims that the \isi{construct state} is \enquote{not a primary linguistic notion}.} 

Basically, the category of state encodes the \concept{syntactic valency} of a noun, i.e.\ whether it must be followed by a complement or not.\footnote{Semantically, this complement may be conceived as a mandatory argument of some sort of the noun, or as an adjunct qualifying the noun, but this distinction is neutralised syntactically. Of course, from a syntactic view point, a mandatory adjunct is \textit{de facto} an argument. } 

It is instructive to contrast this phenomenon with the notion of \concept{semantic valency} of nouns, i.e.\ the number of argument they have in their semantic structure. It is a well known fact of language that some nouns (like \textit{man}) can appear by themselves, while others (e.g.\ \textit{son}) conceptually require some specification. \citet[8]{BarkerPossessive} calls the second group \concept{relational nouns} and the first  \concept{non-relational nouns}. The relational nouns are particular in that they denote relations over pairs of referents, while the non-relational denote simple referents \citep[see also][216f.]{NikolaevaSpencer}. In terms of \isi{valency}, the relational nouns are semantically bi-valent, while the non-relational are semantically mono-valent (counting the referent of the noun itself as one argument). In many languages, this semantic  difference is not related to any morpho-syntactic category. Other languages do mark the difference. One common possibility cross-linguistically is to encode the difference of \isi{valency} by distinguishing two classes of nouns, namely alienable vs.\ inalienable nouns, the latter representing relational nouns mandatory complemented by a possessor. For instance, in the American Navajo language the root \foreign{-beʾ}{milk} cannot appear by its own, but must be possessed: \foreign{bi-beʾ}{her milk (from her breasts)} or \foreign{ʾa-beʾ}{something's milk} \citep{BickelNicholsWals58}.

In contrast to languages which mark semantic \isi{valency}, which is inherent to the nominal lexeme, \isi{state morphology}, marking syntactic \isi{valency}, encodes \textit{ad-hoc} whether in a given context a noun should be understood as relational (i.e.\ requiring a complement) or as potentially non-relational (self-sufficient).\footnote{As the non-relational form is the unmarked form it does not exclude the possibility for a noun to receive a complement. Indeed, inherently relational nouns, such as kinship terms, may still appear in the non-relational form in languages with \isi{state morphology}. Note also that in \ili{Semitic} languages the \cst* applies as well to other nominal elements (such as adjectives or numerals) when they are mandatory accompanied by a complement. It is important to stress  that in such a system inherently relational nouns (inalienable nouns) are not distinguishable morpho-semantically from non-relational nouns.  This is clearly stated by \citet{PatElInalienable} for \BHeb. The claim to the contrary of \citet{SiloniInalienable} regarding the existence of a syntactic class of inalienable nouns in \MHeb is factually wrong, as their alluded ungrammatical or infelicitous examples are neither ungrammatical nor infelicitous.}  The former is marked by the \concept{construct state}, while the latter is marked by the \concept{free state},\footnote{Some authors use the traditional term \concept{absolute state} as opposed to \isi{construct state}. Apart from being less self-explanatory, this term is problematic in the context of Aramaic, as will be explained below.} which is typically also the citation form of a noun.\footnote{A similar proposal, cast in a more formal apparatus, is given by \citet{HellerPossesion}. Heller sees \isi{construct state} nouns as denoting functions from individuals to individuals, in contrast to \isi{free state} nouns which denotes individuals.} By way of analogy, the \isi{construct state} is the nominal parallel of causative morphology in verbs: both add one syntactic argument to the argument structure of their host.

In light of the above, it is clear why the state category is non-projecting. In contrast to case, which signals what kind of dependent a noun is, and therefore should be accessible by constituents outside the NP, the \isi{state morphology} determines whether a nominal governs another nominal NP-internally, and therefore is invisible outside the domain of the NP. Intrinsically, \isi{state morphology} is a head-marking device. 

Thus defined, it is clear that the \isi{construct state}, or rather \isi{state morphology}, is not a phenomenon restricted to \ili{Semitic} languages. In this vein, \citet[74]{CreisselsConstruct} proposes to use the notion of \concept{construct form} \blockquote{as a general label for noun forms that are obligatory in combination with certain types of noun dependents and cannot be analyzed as instances of cross-referencing in the genitive
construction}.\footnote{\citeauthor{CreisselsConstruct} prefers the term \concept{construct form} over construct state due to the confusion arising from the use of the former as a construction label. I shall stick to the traditional term, but note that the notion of \concept{construct state} can relate both to the morphological marking, and to the syntactic position of a \prim (not necessarily marked as such). When in doubt, I will use the term \enquote{construct state marking} or \enquote{construct state form}.} Such a definition equates the \isi{construct state} with the primary relation marker of \citet{PlankIntro}.\footnote{An alternative term, proposed by \citet[268]{DixonBasicII} is \concept{pertensive}  \enquote{based on the \ili{Latin} verb \textit{pertinēre} ‘to belong’}. The term has not gained wide usage, as far as I am aware of. 
Dixon uses this term, moreover, as designating both simple markers and pronominal  markers. It may be for this reason that he does not simply adopt the notion of \cst*, although he is aware of the partial equivalence between the two \citep[310, fn.\ 16.2]{DixonBasicII}. I am grateful to \name{Adam}{Pospíšil}, who drew my attention to this term.}  \citeauthor{CreisselsConstruct} goes on to identify \isi{construct state} forms in a variety of African languages, ranging from \ili{Nilotic languages} in the east to \ili{Wolof} in the west. 

A terminological word of caution is appropriate here. Notwithstanding the above conception of the \cst* notion, it should be noted that in grammars of pre-modern Aramaic a three-way state distinction is given, opposing \concept{absolute state}, \concept{emphatic state} and \concept{construct state}. Both the \isi{absolute state} and the \isi{emphatic state} are in fact instances of the \isi{free state}, as defined above, and the opposition between them is related to the domain of determination: In the earliest stage of Aramaic, the \isi{emphatic state} was used to mark nouns as definite (as early forms of Aramaic lack a syntactic \isi{definite article}; see \sref{ss:intro_NPstructure}),\footnote{For possible origins of the \isi{emphatic state}, unique to Aramaic among \ili{Semitic} languages, see \citet{KönigEmphatic}.} while the \isi{absolute state} was in general used to mark nouns as indefinite. In this setting, the three-way state distinction is justified, in that a \isi{construct state} noun is by itself not determined, but rather the entire CSC inherits its determination feature from its \secn (see \sref{ss:CSCdet}). With time, the definite value of the \isi{emphatic state} was eroded, and it became the default form of the free noun, the \isi{absolute state} being restricted to specific syntactic contexts \parencites{JastrowDetermination}[22, \S 18]{MuraokaSyriac}. 


\subsection{The construct state construction across Semitic languages} \label{ss:cst_Semitic}

From the above discussion it should be  clear that the CSC is essentially a head-marking construction\is{head marking},\footnote{In the \ili{Semitic} languages which mark case, namely \ili{Akkadian} and \CArab, it is a double-marked construction\is{double marking}; see discussion further down in this section.} as in the following \ili{Hebrew} example (contrast with \foreign{\texthebrew{בַּיִת} bayiṯ}{house.\free}):

\hebacex[\BHeb]
{Noun}{Noun (Head-marked AC)}{heb_a}
{בֵּית הַמֶּלֶךְ}
{bēṯ ham-mɛlɛḵ}
{house.\textbf{\cst} \definite-king}
{the house of the king}{}

In \ili{Hebrew} the \isi{construct state} nominals are characterized by \enquote{lighter vocalisation} in comparison with their corresponding \isi{free state}. Sometimes they are marked by specific suffixes, namely \transc{-aṯ} for \fem* \sg* nouns (in contrast to free form \transc{-ā}) and \transc{-ē} (\MHeb \transc{-ey}) for \masc* \pl* nouns (in contrast with \transc{-īm}). All these changes can be  explained by the  \prim losing word stress, and forming one phonological word with the \secn \citep[580]{HeckeConstruct}. This explanation, however,  is diachronic, as in Modern \ili{Hebrew} these forms prevail even when the \prim gets its own stress, as is evident from cases where two \cst* \prims are conjoined:\footnote{In \BHeb it is generally accepted that only one nominal can occur as \prim, as the counter-examples are extremely rare \citep[210]{VerhejGenitive}: \citet[433, \S 128a, note 1]{Gesenius} lists 4 such tentative cases, of which only one is really clear (Ezekiel 31:16): \foreign{\texthebrew{מִבְחַר וְטֽוֹב־לְבָנוֹן} [miḇḥar wə tō̱ḇ] ləḇānōn}{the choice and best of Lebanon} (King James translation). Yet  one finds other cases of intervening material between a \cst* \prim and its \secn; see \citet{FreedmanBroken}. In \Syr too there is a rare occurrence of conjoined \prims; see \example{1009}. Similar examples are attested in Standard \Arab \citep[138f.]{BadawiCarter}.} 

\hebacex{Conjoined Nouns}{Noun Phrase}{MHeb_conj}
{מורֵי ותלמידֵי בֵית הספר}
{[mor-ey ve\cb{} talmid-ey] [bet ha-sefer]}
{teacher-\mpl.\cst{} and\cb{} pupil-\mpl.\cst{} house.\cst{} \definite-book}
{teachers and pupils of the school}{}

\BHeb allows not only nouns as \secns, but also other elements, such as prepositional phrases and clauses \citep[236]{CohenAttribute}.

The CSC of Pre-modern Aramaic is similar in essence to \ili{Hebrew}. The AC system of \ili{Syriac} shall  be treated in detail in \ref{ch:Syriac}.

\largerpage
In \ili{Akkadian} and \CArab, which manifest the old \ili{Semitic} case system, the \secn is further marked by the \isi{genitive case}, giving effectively rise to a doubly marked construction. In \ili{Akkadian}, the \cst* is created by removing the \concept{mimation}, i.e.\ an \transc{-m}\,\~\,\transc{-n} suffix, typical of \free* nouns.
 In some texts, the \sg* \cst* forms are further characterized by losing the case endings,  though they may surface before pronominal \secns \citep[144]{GoldenbergSemitic}.\is{double marking}

\acex[\Akk]{Noun}{Noun (Double-marked AC)}{Akk0}
{bīt-\zero{} awīl-i-m}
{house-\cst{} man-\gen-\free}
{man's house} 
{GoldenbergSemitic}{232}

Clausal \secns in \ili{Akkadian} are marked by a special verbal form, the \concept{subjunctive}:

\acex[\Akk]{Noun}{Clause (Double-marked AC)}{Akk_cl}
{bīt-\zero{} īpuš-u}
{house-\cst{} made.3\masc-\subj}
{house (which) he made}
{CohenNucleus}{80}

The situation in \CArab is similar to \ili{Akkadian}, in that the \concept{nunation} (from \ili{Arabic} \transc{\textarabic{تنوين} tanwīn}), or \transc{-n} suffix, disappears, while case endings, however, are retained.\footnote{I refer here to the functional similarity between \ili{Arabic} and \ili{Akkadian}. Whether the two are historically related is of course a separate question. It is worthwhile noting, in this respect, that also \ili{Hebrew} \free.\mpl\ suffix \transc{-im}, as well as Aramaic \abs.\mpl\ suffix \transc{-in} lose the \phonemic{m} or \phonemic{n} segments respectively in \cst*. \textit{Prima facie}, it seems reasonable to assume that all these functionally and phonetically elements share a common origin.}
 In \ili{Arabic}, the nunation occurs in \isi{complementary distribution} with \isi{definite article}, and therefore is normally seen as an exponent of the indefinite. This analysis, however, is challenged by \citet[91--94]{LyonsDefinitness}. He argues that the nunation (in a variant form), can co-occur with the \isi{definite article} in \pl* and dual nouns, and thus cannot mark indefiniteness. While he analyses it as \textquote[{\cite[93f.]{LyonsDefinitness}}]{a semantically empty marker of nominality}, he notes that it is always dropped in the \cst*. Thus, it seems reasonable to conclude that the nunation is a marker of the \isi{free state}.\footnote{A similar position is maintained by \citet{RetsoPlural}, who investigates also the origin of this system.} The lack of nunation of definite \sg* nouns may be then tentatively explained as resulting from the principle of \concept{economy}, as the \prim of the CSC cannot in general be determined by a \isi{definite article}.\footnote{The exception for this is the CSC headed by adjectives, a construction termed in \ili{Arabic} Grammar \concept{impure annexation}. See \citet[204ff.]{GoldenbergAdjectivization} for further details and analysis.} Conversely, the absence of nunation coupled with the absence of a \isi{definite article} is a clear indicator of the \cst*, as in the following example:\footnote{The situation, however, is complicated by the fact that a certain class of nouns is never marked by the nunation.}\is{double marking}


 
\arabex{Noun}{Noun (Double-marked AC)}{Arab0}
{بيتُ الملكِ}
{bayt-u-{\zero} l-malik-{i}}
{house-\textsc{nom}-{\cst} \textsc{def}-king-{\textsc{gen}}}
{The house of the king.}

In Modern \ili{Arabic} dialects, both the case endings and the nunation are gone, giving rise to pure \isi{juxtaposition} of the \prim and \secn, the only indicator of the CSC being the lack of \isi{definiteness} marking on the \prim:

\acex[\Iraq]{Noun}{Noun (Juxtaposition)}{2024} 
{bēt ʿali}
{house A.}
{Ali's house}
{ErwinIraqi}{370}

\acex[\Malt]{Noun}{Noun (Juxtaposition)}{Malt0} 
{omm Pawlu} 
{mother P.} 
{Paul's mother} 
{Fabri1996}{230}

A remnant of the \cst* marking is however found in \fem* nouns, which in \CArab are written with a \transc{tāʾ marbūṭa} letter (\textarabic{ة}) word-finally. This letter represents a \phonemic{t} phoneme, which is however not pronounced at the edge of a phonological word. In the CSC, such a \prim forms one phonological word with the \secn, and ends therefore with an \transc{-(a)t} segment, effectively marking the \cst* in opposition to the \free* ending \transc{-a}.

\acex[\Iraq]{Noun}{Noun (Head-marked AC)}{2025} 
{sayyār-at ʿali}
{car-\fem.\cst{} A.}
{Ali's car}
{ErwinIraqi}{370}

\acex[\Malt]{Noun}{Noun (Head-marked AC)}{Malt1} 
{nann-t Pawlu}
{grandmother-\fem.\cst{} P.}
{Paul's grandmother'} 
{Fabri1996}{232}

\acex[\Morc]{Noun}{Noun (Head-marked AC)}{Morc1}
{mədras-t Nadya}
{school-\fem.\cst{} N.}
{Nadia's school'} 
{BenmamounConstruct}{479}

\subsection{The construct state construction and determination} \label{ss:CSCdet}

From the above examples, an important characterisation of the classical CSC is apparent, namely the impossibility to mark the \isi{definiteness} feature on the \prim. Instead, the entire NP represented by the CSC acquires its \isi{definiteness} feature from the marking on the \secn \citep[see][587f.]{DoronConstruct}. If a CSC is embedded within another one, only the very last \secn can be marked for \isi{definiteness}, implying \isi{definiteness} for the entire CSC:

\hebacex{Noun}{Noun Phrase (embedded CSC)}{heb0}
{לשכת נְשיא המדינה}
{liška-t nəsiʾ ha-mədina}
{office-\fem.\cst{} president.\cst{} \defi-state}
{the office of the president of the state}{}

In formal linguistic literature, this  phenomenon is referred to using the term \concept{(in)\isi{definiteness} spreading}, as if the \isi{definiteness} marking \enquote{spreads} from the \secn to the entire CSC \citep[cf.][]{DanonDefiniteness}. From a constructionist point of view one can argue that the CSC has only one available slot for marking the \isi{definiteness}, this slot being tied to the \secn. The marking, however, bears on the entire CSC and not on the \secn directly. Such a view is especially fortunate for cases where a marking of \isi{definiteness} on the \secn entails \isi{definiteness} of the entire construction, but not of the \secn itself. This is particularly the case when the \secn is understood non-referentially,  as in the following example, where \foreign{kala}{bride} does not refer to any particular bride, while the entire expression may refer to a specific wedding gown:\footnote{\citet{DanonDefiniteness} brings further converse examples  where a definite-marked \secn entails \isi{definiteness} \emph{only} on the \secn. In such cases, the \isi{definiteness} marking may be assumed to be local to the \secn, the CSC being ambiguous between full-scope and \secn only \isi{definiteness}. Note, however, that syntactically, any \isi{definiteness} marking on the \secn triggers definite agreement of adjectives with the CSC.}    

\hebacex{Noun}{Noun}{heb1}
{שמלת הכלה}
{simla-t ha-kala}
{gown-\fem.\cst{} \defi-bride}
{the wedding gown}{}

As \citet{VerhejGenitive} notes, whenever several conjoined nouns appear as the \secn, they must agree in \isi{definiteness} in order to produce a felicitous CSC.\footnote{Verhej's observation regards Biblical \ili{Hebrew}, but it is by and large valid for Modern \ili{Hebrew} as well.} This again shows that the marking of \isi{definiteness} on the \secn is rather mechanical, and it relates to the marking of \isi{definiteness} of the entire construction.



From the discussion in \sref{ss:state}, we see that there is nothing in definition of the \cst* given there that entails the lack of \isi{definiteness} marking on the same noun. Rather, the situation of the  CSC is comparable to the \isi{complementary distribution} of \isi{determiners} and genitives in other languages (such as \ili{English}: \textit{[the president]'s office}). A possible explanation of this cross-linguistic phenomenon, based upon the linguistic principle of \isi{economy}, is given by \citet{HaspelmathArticle}. 

In some modern \ili{Semitic} languages the situation is somewhat different: While generally there is still only one slot for marking \isi{definiteness}, its position is sometimes changed. 
This is especially clear  in colloquial Modern \ili{Hebrew}, where the \isi{definiteness} marking of the CSC appears regularly before the \prim, especially when the \secn is non referential. In such cases, one may see the CSC as comparable to morphological compounding \parencites{Borer2008}[see also][]{GutmanCompounds}. Thus, \examplebelow{heb2} has the same meaning as \exampleabove{heb1}, the difference being mainly in register. Note, however, that while the \transl{bride} of \examplenp{heb1} could be understood as referential to a specific bride, this is impossible in the following example.

\hebacex{Noun}{Noun}{heb2}
{השמלת כלה}
{ha-simla-t kala}
{\defi-gown-\fem.\cst{} bride}
{the wedding gown}{}









In \ili{NENA} as well, one finds sporadic cases where the determiner appears before the \prim, but without implying a compound reading.

\acex[\JZax]
{Noun}{Noun}{396}
{ʾō šūl nāṭōr-e}
{\defi.\masc{} affair.\cst{} guard(s)-\poss.3\masc}
{the affair of his guards}
{CohenZakho}{121 (134)} 

In modern \ili{Arabic} vernaculars, on the other hand, the \isi{definiteness} marking is still regularly maintained on the \secn:

\acex[\Iraq]
{Noun}{Noun}{2026}
{wuṣḷ-at l-iqmaaš}
{piece-\fem.\cst{} \defi-cloth}
{the piece of cloth}
{ErwinIraqi}{371}

\subsection{The analytic linker construction} \label{ss:Analytic_AC}

As an alternative to the CSC, virtually all \ili{Semitic} languages allow for an  alternative \isi{attributive construction}, which I shall term the \concept{analytic linker construction} (=ALC) or simply the \textsc{linker construction}.\footnote{In the literature, this construction is sometimes termed \concept{analytic genitive construction} \citep[see \textit{inter alia}][]{Grassi2013, BulakhNotaGenitivi}. The term \concept{genitive construction} should be understood in this context as equivalent to the term \concept{attributive construction}, as no \isi{genitive case} marking is necessary implied. \label{fn:ALC}} The essence of this construction is that the \prim is in the \free*, while the \secn is following a third element with which it forms a syntactic (but not morphological) co-constituent. I shall term this element \concept{linker} (glossed \lnk), in resonance with Plank's \enquote{link}, with some reserves regarding this terminology below. 

Thus, in Modern \ili{Hebrew}, the analytic alternative to \example{heb1} is the following:

\hebacex{Noun}{Noun}{heb5}
{הבַית של המלך}
{ha-bayit [šel ha-meleḵ]}
{\defi-house.\free{} \lnk{} \defi-king.\free}
{the house of the king}{}

The \lnk* is treated in the literature as a {preposition} or as a {genitive marker}\is{genitive marking} \citep[a.k.a.\ \emph{\isi{nota genitivi}}, see][]{BulakhNotaGenitivi}, or sometimes both \citep[cf.\ the \emph{\isi{genitive preposition}} of][582]{DoronConstruct}. In fact, as \citet[3--6]{GoldenbergAttribution} claims, it is best treated cross-Semitically\il{Semitic} as a pronominal element being notionally in \cst*, and capable of  standing in \isi{apposition} with an optional explicit nominal antecedent being in \free*.\footnote{As we shall see, there are exceptions to this rule, such as the rare \Syr \example{1045} or more systematically in \JUrm; see \sref{ss:JUrm_cst_lnk}.} This is represented schematically as follows:

 $$[\opt{X_{\free}} \leftrightarrow [\lnk\ \mapsto Y]\sub{CSC}]\sub{ALC}$$
 
 Note that the \lnk*, being a pronoun heading a CSC, is quite special in that it acts as a head of a complex NP, in contrast to most pronouns which replace an entire NP.
 

 
 From a diachronic view point, the \lnk*s of many \ili{Semitic} languages are in fact cognate with the \ili{Akkadian} \cst* pronoun \transc{ša}:\footnote{The \ili{Hebrew} \lnk* \transc{šel}, present in \example{heb5} is in fact particular in that it has  incorporated  the preposition \foreign{l-}{to} \citep[240]{GoldenbergSemitic}. } 

\acex[\Akk]{Pronoun}{Noun}{Akk3}
{ša šarr-i-m}
{\pro.\cst{} king-\gen-\free{}}
{that of the king} 
{GoldenbergSemitic}{232}



The term \concept{linker} may seem unfortunate for an element which can serve as an independent syntactic head. Note, however, that even when no primary is explicitly present, the \lnk* mediates between an understood primary and a necessarily present secondary. Moreover, from the point of view of discourse frequency, more often than not it does link between two explicit nominal elements, bleaching its pronominal value and rendering it rather a construction marker. When necessary, I shall  differentiate between a \concept{pronominal linker}, capable of standing by its own without a primary, akin to \example{Akk3}, and a \concept{pure linker}, necessarily standing between two elements, being effectively a simple \secn marker, similarly to the \ili{English} preposition \transl{of} used in the possessive sense.

The question of the different semantics of the ALC and the CSC has been much researched in the literature of \ili{Semitic} languages (for \MHeb see for instance \cite{ShlezingerRavid1998} and bibliography there). The exact functional difference is outside the scope of this work, and I shall only briefly touch this question regarding the languages under study.


\subsection[Goldenberg's typology of ACs in Semitic]{Goldenberg's typology of attributive constructions in Semitic} \label{ss:Goldenberg_typology}

\citet[Ch.\ 14]{GoldenbergSemitic} presents an elaborate typology of ACs in \ili{Semitic} languages. Following his previous works \citep{GoldenbergAttribution}, he sees the CSC (the \concept{genitive construction} in his terminology) as the basic exponent of the \isi{attributive relationship} in \ili{Semitic} languages. His classification is based first and foremost on the important observation that the \isi{attributive relationship} is not restricted to  nouns, but can in fact hold also between other phrasal categories. Thus, the \secn (attribute) can be a noun, a pronoun, a \isi{prepositional phrase} (PP) or a clause, while the \prim (head) can be a noun or a pronoun (and in fact also \isi{adverbial} elements, namely prepositions or conjunctions). The various combinations yield 8 different patterns, presented in \vref{tb:Goldenberg_pos}.

 
\begin{table}[h!]
\centering
\begin{tabular}{l l l}
\toprule
	& Head 		& Attribute \\
\midrule 
A   & Noun		& Noun \\
B	& Noun		& Pronoun \\
C	& Pronoun	& Noun \\
E	& Pronoun	& Pronoun \\
G	& Noun		& PP \\
H	& Pronoun	& PP \\
I	& Noun		& Clause \\
J	& Pronoun	& Clause \\
\bottomrule
\end{tabular}

\caption{Members of the attributive relationship \citep[Ch.\ 14]{GoldenbergSemitic}} \label{tb:Goldenberg_pos}

\end{table}

Syntactically, all these patterns can in principle be expressed by the CSC. Yet,  when a pronoun is involved, they may (or sometimes must) be expressed morphologically. For instance, Pattern B is normally expressed by attaching a possessive \isi{pronominal suffix} to the \isi{head noun}, yielding a morphological construction  somewhat different from the CSC. Moreover, adjectives, according to Goldenberg, are simply morphological realisations of pattern C, where a pronoun, denoting a referent, and a nominal attribute denoting a quality, are fused together into one word. 

Pronominal elements play  a further important role in Goldenberg's classification, since they permit the extension of the basic AC, be it the syntactic CSC or a morphological construction, into more elaborate periphrastic constructions. This is possible, since the pronominal elements can stand in \emph{apposition} (in the sense defined in \sref{ss:threeRel}) with other NPs. For instance, \example{Akk3}, being an instance of Goldenberg's Pattern C, can be extended by adding a nominal \prim appositional to the \isi{pronominal head} of the CSC, yielding the ALC.

\acex[\Akk]
{Noun$\leftrightarrow$Pronoun}{Noun}
{Akk1}
{mārum [ša šarr-i-m]}
{son \pro.\cst{} king-\gen-\free{}}
{the king's son}
{GoldenbergSemitic}{232}


Goldenberg's analysis of the ALC in \Syr is given in \sref{ss:syr_ALC}, while a more elaborate extension, the Double Annexation Construction, involving two appositions, is presented in \sref{ss:syr_DAC} in the context of \Syr. 

































Goldenberg's pronominal elements are quite similar to Plank's relatedness-indicators. Yet  their definitory property is that they are pronouns, i.e.\ they substitute a noun in an AC, and as such they can form an independent NP constituent together with their antipodal \isi{locus}. Inflectional properties reflecting number, gender, person, or case of a co-referenced noun are incidental and do not need to appear. For instance, the \Akk \cst* pronoun \transc{ša}, shown in \example{Akk1}, does not inflect.  

\section{Typology of attributive constructions used in this study} \label{ss:typology_here}

The typology of attributive constructions used in this study is informed both by Plank's typology (see \ref{ss:Plank_typology}) and Goldenberg's typology (see \ref{ss:Goldenberg_typology}), while being adapted to the languages studied, namely the \ili{NENA} dialects.

The classification of ACs undertaken in this study is based, on the one hand, on the morphemic make-up of the constructions (syntagmatic axis) and, on the other hand, on the categorical variation available for the \prims and \secns (paradigmatic axis). The two axes are detailed in the two sections below. 

\subsection{Syntagmatic axis} \label{ss:synt_axis}

I distinguish between two \textsc{loci}\is{locus} of marking (\prim and \secn) and two types of marking: simple relation markers and pronominal markers\is{pronominal suffix} (the latter named \isi{relatedness-indicators} in Plank's terminology\ia{Plank, Frans@Plank, Frans}). Ignoring the possible accumulation of markers on one locus, and leaving aside the question of the ordering of the elements, this leads to 7 principal  constructions, summarized in \ref{tb:main_AC_strategies}. 

 Following \citet[39]{PlankIntro}, $X$ represents the \prim, $Y$ the \secn, while $x$ and $y$ are co-referential pronominal markers. Subscripts represent morphological marking. Curly brackets indicate optional elements, while square brackets delimit independent NPs. 

Further comments about each construction and their usage in the \ili{NENA} group are given below. The terminology and the relevant glossing conventions used f	or each construction are introduced here as well.

\begin{table}[h!]
\centering
\begin{tabular}{l l}
\toprule
Juxtaposition & $X\ Y$ \\

Genitive case marking & $X\ Y_{\gen}$ \\

Construct state construction & $X_{\cst}\ Y_{\opt\gen}$ \\




Juxtaposition-cum-agreement & $X\ Y_{\agr}$ \\

Analytic linker construction & $\opt{X_{\opt\cst}} [x_{\lnk}\ Y_{\opt{\gen}}]$ \\

Possessive suffix marking & $[X-y_{\poss}]\  \opt{Y_{\opt\gen}}$ \\

Double \isi{annexation} construction & $[X-y_{\poss}]\  [x_{\lnk}\ Y_{\opt\gen}]$ \\

\bottomrule
\end{tabular}
\caption{Principal Attributive Constructions}\label{tb:main_AC_strategies}
\end{table}




\begin{description}





\item[Juxtaposition] The \zero-marking strategy: the two  members of the construction are merely juxtaposed to each other.








\item[Simple \prim marking] The \prim is marked morphologically by the \cst* (glossed \cst), yielding the \concept{construct state construction} (CSC). In the \ili{NENA} dialects, I shall differentiate between three types of \cst*: a \enquote{classical} \cst* characterized by phonological reduction of the corresponding \isi{free state}, typically \isi{apocope} of the last vowel; a suffixed marker \ed, typically replacing the last vowel; and a suffixed marker originating in the \ili{Iranic} \ez*.\footnote{In the interlinear glossing of examples the two first types shall be differentiated as .\cst\ vs.\ -\cst, while the \ez* shall be glossed as -\ez. In abstract representations of constructions, however, I shall use the gloss .\cst\ to encompass all three types. \label{ft:cst_glossing}}

\item[Simple \secn marking] The \secn is marked morphologically by means of \concept{genitive case} (glossed \gen). In \ili{NENA} dialects only some \isi{determiners} can be marked by \gen* case. Alternatively, in \Syr one finds a \isi{dative preposition} (glossed \dat) marking the \secn (see \sref{ss:dat_lnk}). A clausal \secn can be marked as such by means of a \rel* (glossed \rel).

\item[(Simple) \isi{double marking}] Both the \prim and the \secn are marked with the above markers.  As in the \ili{NENA}  dialects \gen* marking is normally only possible with some \secns, I treat this as a variant of the CSC.








\item[Pronominal \secn marking] A pronominal \lnk* (glossed \lnk{}, representing the \prim, intervenes between the two construction members, bonding syntactically with the \secn (see \sref{ss:Analytic_AC}), yielding the \concept{analytic linker construction} (ALC).
Note that the \secn  may additionally be marked by \gen* case. In this construction, an explicit nominal \prim may appear in \free*,  or more rarely in \cst* (see \sref{ss:JUrm_cst_lnk}). If the pronominal element is not an overt linker but is rather fused morphologically with the \secn, as is the case of adjectives according to Goldenberg's analysis, then the \secn exhibits agreement features (marked as \agr); in this case I shall speak of the \concept{juxtaposition-cum-agreement} construction (although it is assimilated with the pure \isi{juxtaposition} construction, as agreement features can be neutralised).

\item[Pronominal \prim marking] The \isi{head noun} is marked by a \isi{pronominal suffix} representing the \secn. Following the traditional terminology in \ili{Semitic} studies, I call these suffixes \concept{possessive suffix}\textsc{es} (glossed \poss), although their usage is wider than denoting possession only.\footnote{Cf.\ \citet[ch.\ 2]{Ornan1964} who uses the \ili{Hebrew} term \foreign{\texthebrew{כינוי קניין} kinuy qinyan}{possessive pronoun}.} In \ili{NENA} this construction occurs without an explicit nominal \secn, the suffix effectively representing the \secn. In other languages, such as \Turk, one finds this marking co-occurring with an explicit nominal \secn (which may or may not  be marked by \gen* case; see \example{Turk1}).
 
\item[Pronominal \isi{double marking}] In this construction both the \prim and the \secn are marked with the above pronominal markers, yielding the \concept{double annexation construction} (DAC).\footnote{In the usage of this term I follow \citet[234, fn. 15]{GoldenbergSemitic}, who credits \citet[124 {[2011: 85]}]{Ornan1964} for introducing this term in \ili{Hebrew} as \transc{\texthebrew{סמיכות כפולה} smixut kfula}. In descriptive grammars of \ili{Hebrew} \citep[e.g.][34]{GlinertGrammar} as well as in typological works referring to \ili{Hebrew} \citep[e.g.][366]{ComrieThompson} it is often translated as \concept{double genitive}. For the use of the term \concept{genitive} in this respect see \vref{fn:ALC}. \label{ft:DAC_term}}
  Such a construction is very rare in \ili{NENA}, but is found in \Syr (see \sref{ss:syr_DAC}). 

\end{description}



In addition to these terms, I shall use occasionally the more general terms \concept{head marking}, \concept{dependent marking} and \concept{double marking}, as explained above in \sref{ss:head_vs_dep}.

\subsection{Paradigmatic axis}

For a fine-grained classification of the ACs, it is profitable to examine the question of which elements can appear as \prims and as \secns apart from nouns. The classification of these elements  is a necessary methodological choice, and does not reflect any cross-linguistic claims regarding the universality of the proposed categories.

My classification is based on the traditional distinction between the Parts-of- Speech (Nouns, Pronouns, Adjectives, Participles, Infinitives, Adverbs). Additionally, I make a distinction between one-word elements and phrasal, multi-word, constituents.\footnote{This distinction permits us to distinguish between constructions which are morphological in nature and require a single-word host, from those which are syntactic. It should not be understood as implying that single-word constituents can not act as phrasal constituents.} 
 I distinguish \concept{CP nouns} [=Complex Predicate Nouns] as a special functional sub-category of nouns   which participate, together with a light verb, in complex predication structures. Complex predication is quite common in \ili{Iranic} languages such as \Per \citep[see for instance][]{SamvelianComplex}, and has been borrowed to some extent into some \ili{NENA} dialects.
On the other hand, I conflate into one category of \concept{adverbial}\textsc{s} all elements which head phrases of \isi{adverbial} function, be they prepositions, conjunctions or adverbs, following \citet{CohenNucleus}.\footnote{The rationale for this choice is that often one and the same element can take all three functions, depending on its complements, such as the \ili{English} word \transl{before}. In the example titles, however, I shall give more precise labels (Preposition, Conjunction, Adverb), unless I wish to emphasize the general \isi{adverbial} nature of the element in question. \label{ft:adverbial}} In the case of the \isi{analytic linker construction}, where no explicit \prim appears besides the pronominal \lnk*, I shall treat this absence as a \concept{zero} (\zero) \prim.

In the \secn position I observe two further categories: ordinal numerals (\transl{first}, \transl{second}, etc.) and clauses. Thus, as possible \prims or \secns I distinguish between the following categories:

\begin{enumerate}

\item Noun \opt{Phrase}

\item Pronoun

\item CP Noun

\item Infinitive \opt{Phrase}

\item Participle \opt{Phrase}

\item Adjective \opt{Phrase}

\item Ordinal

\item Clause

\item Adverbial

\item Zero (\zero)\is{zero}

\end{enumerate}

\subsection{Synopsis}

\begin{table}[th!]
\centering
\begin{tabular}{c|c||c|c|c}
\toprule
\Prim (X) & \cst\ [± \poss] & \lnk & \gen  & \Secn (Y) \\
\midrule






\begin{tabular}{l}
Noun  \\
Pronoun  \\
CP Noun \\
Infinitive  \\
Participle  \\
Adjective  \\
Adverbial \\
Zero  (\zero) \\
\end{tabular}
&
\begin{tabular}{l}
Ap. [± \poss] \\
\ed \\
-\ez  \\
\zero \\
\end{tabular}
&
\begin{tabular}{l}
+ \\
\zero \\
\end{tabular}
&
\begin{tabular}{l}
+ \\
\zero \\
\end{tabular}
&
\begin{tabular}{ll}
Noun  & \\
Pronoun & \\
CP Noun &\\
Infinitive & \\
Participle & \\
Adjective & [± \agr] \\
Ordinal & [± \agr] \\
Adverbial & \\
\multicolumn{2}{l}{[± \rel] Clause} \\
\end{tabular} 
\\

\bottomrule

\end{tabular}
\caption{Parameters of an AC structure, disregarding order variation} \label{tb:AC_parameters}
\end{table}


The different syntagmatic and paradigmatic possibilities for ACs in \ili{NENA}, as analysed in the current work, are summarised in \ref{tb:AC_parameters}. Each column shows the variation available at each morphemic slot. To this one should add the two ordering possibilities: typically the \prim precedes the \secn (X Y), but also the inverse order can be found.

The \prim may be marked by \cst* morphemes of various types: \concept{apocope} (=Ap.), the native suffix \ed, or \ez* marking, or it may stay unmarked (\zero)\is{zero}. Following the apocopate \cst* (or a variant thereof)
 one may find a \concept{possessive suffix}, which functions as a pronominal \secn.

As for the \secn marking, there are two main markers: a pronominal \lnk* and/or a  \gen* case. These may independently be present (+) or absent (\zero).  Adjectives and \isi{ordinals} may show additionally agreement features (\agr), while a \rel* may precede a clausal \secn.

