\setotherlanguage{hebrew}
\setotherlanguage{arabic}
\setotherlanguage{syriac}
\setotherlanguage{french}

%add all your local new commands to this file

\labelformat{chapter}{Chapter #1}
\labelformat{section}{\S #1}
\labelformat{subsection}{\S #1}
\labelformat{subsubsection}{\S #1}
\labelformat{paragraph}{\S #1}
\labelformat{figure}{Figure #1}
\labelformat{table}{Table #1}
\labelformat{footnote}{footnote #1}

\providecommand{\example}[1]{example~\ref{ex:#1} \vpageref{ex:#1}}
\providecommand{\exampleabove}[1]{example~\ref{ex:#1} \vpageref[above]{ex:#1}}
\providecommand{\examplebelow}[1]{example~\ref{ex:#1} \vpageref[below]{ex:#1}}
\providecommand{\examplenp}[1]{example~\ref{ex:#1}}

\providecommand{\Example}[1]{Example~\ref{ex:#1} \vpageref{ex:#1}}
\providecommand{\examples}[2]{examples~\ref{ex:#1}--\ref{ex:#2} \vpageref{ex:#1}}

\newcommand{\sub}[1]{_\textsc{\tiny#1}}

\newcommand{\concept}[1]{\textsc{#1}} 

\newcommand{\defaultDialect}{}

\newcommand{\acex}[9][\defaultDialect]{\protectedex{\ex. #1: \textbf{#2--#3} \label{ex:#4} \nopagebreak \bg.[] #5 \\ #6 \\ \transl{#7}
\ifx\relax#8\relax#9\else\citep[#9]{#8}\fi \par }} % `\MakeUppercase#6'
 
\newcommand{\acexfn}[9][\defaultDialect]{%always puts (i)
\protectedex{\ex.[(i)] #1: \textbf{#2--#3} \label{ex:#4} \nopagebreak \bg.[] #5 \\ #6 \\ \transl{#7}
\ifx\relax#8\relax#9\else\citep[#9]{#8}\fi\par}}

\newcommand{\acexfnii}[9][\defaultDialect]{%always puts (ii)
\protectedex{\ex.[(ii)] #1: \textbf{#2--#3} \label{ex:#4} \nopagebreak \bg.[] #5 \\ #6 \\ \transl{#7}
\ifx\relax#8\relax#9\else\citep[#9]{#8}\fi\par}}
 

% hebrew
\newcommand{\hebacex}[9][\MHeb]{\protectedex{\ex. #1: \textbf{#2--#3} \label{ex:#4} \nopagebreak \b.[] \texthebrew{#5} \cg.[]  #6 \\ #7 \\ \transl{#8} #9 \par }}
% arabic
\newcommand{\arabex}[9][\Arab]{\protectedex{\ex. #1: \textbf{#2--#3} \label{ex:#4} \nopagebreak \b.[] \textarabic{#5} \cg.[]  #6 \\ #7 \\ \transl{#8}  #9 \par }}





\newcommand{\syacex}[9][]{\protectedex{%
\def\temp{#1}\ifx\temp\empty
  \ex.
\else
  \ex.[#1]
\fi 
\Syr: \textbf{#2--#3} \label{ex:#4} \nopagebreak \b.[] \textsyriac{#5} \cg.[] #6 \\ #7 \\ \transl{#8} (#9) \par }}

\newcommand{\Pesh}{\textit{Peshiṭta}}
\newcommand{\apud}{\textit{apud} }

\newcommand{\syr}[1]{\textsyriac{\abjadsyriac{#1}}}

\newcommand{\lex}[7]{
\acex[#1]
{Noun}{Noun}{#2}
{#3}
{#4}
{#5}
{#6}{#7} 
\par 
}



\renewcommand{\d}{\transc{d-}\xspace}
\newcommand{\D}{\transc{d}}
\newcommand{\ed}{\mbox{\transc{-əd}}\xspace}

\newcommand{\vapp}{$\updownarrow$}

\hyphenation{Sprach-ge-schich-te}
\hyphenation{a-ra-mä-isch}
\hyphenation{Ghan-nam}
\hyphenation{Ge-sell-schaft}
\hyphenation{Hil-des-heim}
\hyphenation{Ma-nu-scripts}
\hyphenation{Mor-pho-lo-gy}
\hyphenation{Se-mi-tic}
\hyphenation{Pee-ters}
\hyphenation{La-voi-sier}
\hyphenation{U-ni-ver-si-té}
\hyphenation{Mor-pho-lo-gy}
\hyphenation{Re-prin-ted}
\hyphenation{Je-ru-sa-lem}

%\Urlmuskip=0mu plus 2mu

% \newcommand{\audiosign}{{\fontspec{Arial Unicode MS}♫}}
\newcommand{\audiosign}{{\texttt{♫}}}

\newcommand{\dlnk}{\textit{d}\textsubscript{\lnk}\cb}
\newcommand{\dgen}{\textit{d}\textsubscript{\gen}-}
\newcommand{\ddet}{\textit{d}+\textsc{dem}}

%\setcounter{secnumdepth}{4} % number paragraphes

\newcommand{\N}{\textbf{N} } % for Mutzafi's references
\newcommand{\Q}{\textsc{q}} % in table of diachrony2

\newcommand{\reex}[1]{example \ref{ex:#1bis}=\example{#1}}

% Language definitions were here

\newfontfamily\hebrewfont[Script=Hebrew,ItalicFont=*, Scale=0.9]{SBLHebrew.ttf}
\newfontfamily\arabicfont[Script=Arabic,ItalicFont=*,Scale=1.4]{arabtype.ttf}
\newfontfamily\syriacfont[Script=Syriac]{EstrangeloEdessa.ttf}

%\newcommand{\texthebrew}[1]{{\hebrewfont #1}}
%\newenvironment{hebrew}{\hebrewfont}{}
%\newcommand{\hebrewnumeral}[1]{{\red{#1}}}

%\newcommand{\textsyriac}[1]{{\syriacfont #1}}
%\newenvironment{syriac}{\syriacfont}{}

%\newcommand{\abjadsyriac}[1]{\color{red}{\syriacfont #1}}
%\newcommand{\textfrench}[1]{{\color{red}#1}}

%\newcommand{\textarabic}[1]{{\arabicfont #1}}
% \newenvironment{arabic}{\arabicfont}{}



%  transliteration and transcription in text

\makeatletter

\newcommand{\transc}[1]{\textit{#1}}
\newcommand{\transl}[1]{‘#1’} % better than using \enquote for nesting reasons
\newcommand{\foreign}[2]{\transc{#1} \transl{#2}}
\newcommand{\foreigngloss}[2]{\transc{#1} [#2]}

\newcommand{\phonemic}[1]{/\textit{#1}/}
\newcommand{{\phonetic}[1]}{[{{#1}}]}  % used to be in italics
\def\ph/#1/{\phonemic{#1}}



\newcommand{\prim}{primary\xspace}
\newcommand{\Prim}{Primary\xspace}
\newcommand{\prims}{primaries\xspace}
\newcommand{\Prims}{Primaries\xspace}
\newcommand{\secn}{secondary\xspace}
\newcommand{\Secn}{Secondary\xspace}
\newcommand{\secns}{secondaries\xspace}
\newcommand{\Secns}{Secondaries\xspace}


%persons

\newcommand{\first}{1\textsuperscript{st}\xspace}
\newcommand{\second}{2\textsuperscript{nd}\xspace}
\newcommand{\third}{3\textsuperscript{rd}\xspace}
\renewcommand{\th}{\textsuperscript{th}\xspace}

% varia

\newcommand{\opt}[1]{\{#1\}} % optionality marking
\newcommand{\zero}{∅} %{$\emptyset$} % zero morpheme
\newcommand{\agr}{\textsc{agr}} % agreement features

% NP markers 
\DeclareRobustCommand{\cst}{\@ifstar{construct state}{\textsc{cst}}} % Construct state
\DeclareRobustCommand{\free}{\@ifstar{free state}{\textsc{free}}} % Free (non-construct) state
\DeclareRobustCommand{\emp}{\@ifstar{emphatic state}{\textsc{emph}}} % emphatic state
\DeclareRobustCommand{\abs}{\@ifstar{absolute state}{\textsc{abs}}} % absolute state
\renewcommand{\schwa}{V\textsubscript{\cst}} %\textsc{schwa}} % stranded schwa


\DeclareRobustCommand{\ez}{\@ifstar{Eza\-fe}{\textsc{ez}}} % Ezafe (or Izafe?)
\DeclareRobustCommand{\gen}{\@ifstar{genitive}{\textsc{gen}}} % Genitive case
\DeclareRobustCommand{\lnk}{\@ifstar{linker}{\textsc{lnk}}} % Linker/linking pronoun
\DeclareRobustCommand{\comp}{\@ifstar{complementizer}{\textsc{comp}}} % complementizer
\DeclareRobustCommand{\nmlz}{\@ifstar{nominalizer}{\textsc{nmlz}}} % complementizer
\newcommand{\lnkcomp}{\comp} %{\lnk\textsubscript{\textsc{nmlz}}} % linker acter as nominalizer
\newcommand{\lnkd}{\lnk} % {\textsc{lnk.cst}} % Linker marked as construct state: /did/ (to consider if we keep this notation of JZax)
%\DeclareRobustCommand{\poss}{\@ifstar{possessive pronominal suffix}{\textsc{poss}}}  % possesive pronoun/suffix
\newcommand{\poss}{\textsc{poss}}
\newcommand{\pro}{\@ifstar{pronoun}{\textsc{pro}}} % generic gloss for pronouns

\DeclareRobustCommand{\acc}{\@ifstar{accusative}{\textsc{acc}}}
\DeclareRobustCommand{\nom}{\@ifstar{nominative}{\textsc{nom}}}
\newcommand{\dat}{\textsc{dat}} %{\@ifstar{dative}{\textsc{dat}}}
\DeclareRobustCommand{\obl}{\@ifstar{oblique}{\textsc{obl}}}

% Gender
\DeclareRobustCommand{\fem}{\@ifstar{feminine}{\textsc{fs}}}  % Feminine singular
\DeclareRobustCommand{\masc}{\@ifstar{masculine}{\textsc{ms}}} % Masculine singular



% Number
\DeclareRobustCommand{\sg}{\@ifstar{singular}{\textsc{sg}}} % Singular
\DeclareRobustCommand{\pl}{\@ifstar{plural}{\textsc{pl}}} % Plural

% Number cum gender
\DeclareRobustCommand{\fpl}{\@ifstar{feminine plural}{\textsc{f.pl}}}  % Feminine pluaral
\DeclareRobustCommand{\mpl}{\@ifstar{masculine plural}{\textsc{m.pl}}} % Masculine pluaral


\newcommand{\adj}{\textsc{adj}} % adjectival derivation

% Invariable adjectives

\newcommand{\invar}{\textsc{inv}}

\newcommand{\ord}{\textsc{ord}} % Ordinal
% definitness

\newcommand{\definite}{\textsc{def}} % Definite (note \def is a TeX command)
\newcommand{\defi}{\definite} %short cut
\newcommand{\indef}{\textsc{indf}}

% decide when to use \definite and when \dem

% demonstrative pronouns

\DeclareRobustCommand{\dem}{\@ifstar{demonstrative pronoun}{\textsc{dem}}}
\newcommand{\far}{\textsc{dist}}
\newcommand{\near}{\textsc{prox}}

% Pronouns
\newcommand{\refl}{\textsc{refl}} % Reflexive

% Verbal Phrase

\renewcommand{\neg}{\textsc{neg}} % Negator (notice normally \neg gives the math symbol)



\newcommand{\cop}{\textsc{cop}} % Copula
\newcommand{\exist}{\textsc{ex}} %{$\exists$} %∃ - existential copula

\DeclareRobustCommand{\rel}{\@ifstar{relativizer}{\textsc{rel}}} % relativizer (/ke/, /ya/ etc.)

\newcommand{\patient}{\textsc{p}} % object marker
\newcommand{\agent}{\textsc{a}} % subject marker


% TAM glosses
\newcommand{\sbjv}{\textsc{sbjv}} % subjunctive
\newcommand{\subj}{\sbjv}  % subjunctive (alternative command)
%\newcommand{\pres}{\textsc{prs}} % present base
\newcommand{\prf}{\textsc{prf}} % Perfect (also for preterite)
\newcommand{\iprf}{\textsc{impf}} % Imperfect 
\newcommand{\resl}{\textsc{res}} % Resultative particle
\newcommand{\pst}{\textsc{pst}} % Past marker (for =wa)
\newcommand{\ind}{\textsc{ind}} % Indicative (for k-)
\newcommand{\fut}{\textsc{fut}} % Future (for b-)
\renewcommand{\inf}{\textsc{inf}} % Infinitive
\newcommand{\ptcp}{\textsc{ptcp}} % Participle
\newcommand{\prtc}{\ptcp} % Participle (alternative command)
\newcommand{\imp}{\textsc{imp}} % imperative
\newcommand{\prog}{\textsc{prog}} % progressive / habitual (Kurdish (d)a) <- but this should be indicative
\newcommand{\pass}{\textsc{pass}} % passive
\newcommand{\aux}{\textsc{aux}} % auxiliary

\newcommand{\dir}{\textsc{dir}} % directional
% Some conventions: 
% Present base is encoded by using the present form of the English verb
% If the /k-/ appears use \ind
% If it is marked irralis use \subj	
% Past base - the past form of the English verb
% Perfect/Resultative - the past form + \res

\newcommand{\caus}{\textsc{caus}} % causative
\newcommand{\compr}{\textsc{compr}} % comperative
\DeclareRobustCommand{\supr}{\@ifstar{superlative}{\textsc{super}}}

% Misc

%Tilde symbol
\renewcommand{\~}{\hspace*{-0.17ex}\raise.17ex\hbox{\bfseries\,\textasciitilde}}

%= symbol used as clitic marker - same length as hyphen



\renewcommand{\cb}{% clicic boundary
  \settowidth{\@tempdima}{-}% Width of hyphen
  \resizebox{\@tempdima}{\height}{=}%
}

%\renewcommand{\=}{\cb}

\renewcommand{\>}{%
  \settowidth{\@tempdima}{-}% Width of hyphen
  \resizebox{\@tempdima}{\height}{>}%
}


\makeatother

\newcommand{\+}{\textnormal{⁺}} % superscript plus
\newcommand{\parplus}{\textnormal{⁽⁺⁾}}

\newcommand{\PP}{Prepositional Phrase}

% List of dialects/languages

% \DeclareRobustCommand{\JSan}{\@ifstar{Sandandaj\il{NENA, Sanandaj (Jewish)}\xspace}{JSanandaj\il{NENA, Sanandaj (Jewish)}\xspace}}
% \DeclareRobustCommand{\JZax}{\@ifstar{Zakho\il{NENA, Zakho (Jewish)}\xspace}{JZakho\il{NENA, Zakho (Jewish)}\xspace}}
% \DeclareRobustCommand{\JUrm}{\@ifstar{Urmi\il{NENA, Urmi (Jewish)}\xspace}{JUrmi\il{NENA, Urmi (Jewish)}\xspace}}
\DeclareRobustCommand{\JSan}{JSanandaj\il{NENA!Sanandaj (Jewish)}\xspace}
\DeclareRobustCommand{\JZax}{JZakho\il{NENA!Zakho (Jewish)}\xspace}
\DeclareRobustCommand{\JUrm}{JUrmi\il{NENA!Urmi (Jewish)}\xspace}
\DeclareRobustCommand{\CSan}{CSanandaj\il{NENA!Sanandaj (Christian)}\xspace}
\DeclareRobustCommand{\CUrm}{CUrmi\il{NENA!Urmi (Christian)}\xspace}

\DeclareRobustCommand{\Alq}{Alqosh\il{NENA!Alqosh}\xspace}
\DeclareRobustCommand{\Ank}{ʿAnkawa\il{NENA!Ankawa (ʿAnkawa)}\xspace}
\DeclareRobustCommand{\Barw}{Barwar\il{NENA!Barwar}\xspace}
\DeclareRobustCommand{\Nrt}{Nerwa\il{NENA!Nerwa texts}\xspace}
\newcommand{\NrT}{\Nrt}
\DeclareRobustCommand{\JArb}{Arbel\il{NENA!Arbel}\xspace}
\newcommand{\Arb}{\JArb} % aliasing
\DeclareRobustCommand{\JSul}{Su\-le\-ma\-niy\-ya\il{NENA!Sulemaniyya and Ḥalabja}\xspace} % and Ḥalabja
\DeclareRobustCommand{\Amd}{Amədya\il{NENA!Amədya}\xspace}
\DeclareRobustCommand{\JArd}{JAradhin\il{NENA!Aradhin (Jewish)}\xspace}
\DeclareRobustCommand{\CArd}{CAradhin\il{NENA!Aradhin (Christian)}\xspace}
\DeclareRobustCommand{\Barz}{Barzani\il{NENA!Barzani}\xspace}
\DeclareRobustCommand{\Baz}{Baz\il{NENA!Baz}\xspace} % in Hakkari
\DeclareRobustCommand{\Bes}{Bēṣpən\il{NENA!Bēṣpən}\xspace}
\DeclareRobustCommand{\Betn}{Betanure\il{NENA!Betanure}\xspace}
\DeclareRobustCommand{\Boh}{Bohtan\il{NENA!Bohtan}\xspace}
\DeclareRobustCommand{\Cal}{Challa\il{NENA!Challa}\xspace}
\newcommand{\Diy}{Diyana-Zariwaw\il{NENA!Diyana-Zariwaw}\xspace}
\newcommand{\DiyZ}{Diyana-Z.\il{NENA!Diyana-Zariwaw}} % shorter version to use in tables
\DeclareRobustCommand{\Dohok}{Dohok\il{NENA!Dohok}\xspace}
\DeclareRobustCommand{\Gaz}{Gaznax\il{NENA!Gaznax}\xspace}
\DeclareRobustCommand{\Her}{Hertevin\il{NENA!Hertevin}\xspace}
\DeclareRobustCommand{\Jil}{Jilu\il{NENA!Jilu}\xspace}
\DeclareRobustCommand{\Ker}{Kerend\il{NENA!Kerend}\xspace}
\DeclareRobustCommand{\Khabur}{Khabur\il{NENA!Khabur dialects}\xspace}
\DeclareRobustCommand{\Khanaqin}{Khanaqin\il{NENA!Khanaqin}\xspace}
\DeclareRobustCommand{\JKoy}{Koy Sanjaq\il{NENA!Koy Sanjaq (Jewish)}\xspace} % there is also a C. Koy
\newcommand{\Koy}{\JKoy}
\DeclareRobustCommand{\Qar}{Qaraqosh\il{NENA!Qaraqosh}\xspace} 
\DeclareRobustCommand{\Rus}{Rustaqa\il{NENA!Rustaqa}\xspace} % J.  Rustaqa
\DeclareRobustCommand{\Ruw}{Ruwandiz\il{NENA!Ruwandiz}\xspace} % J. Ruwandiz
\DeclareRobustCommand{\Sandu}{Sandu\il{NENA!Sandu}\xspace}
\DeclareRobustCommand{\Salamas}{Salamas\il{NENA!Salamas}\xspace}
\DeclareRobustCommand{\Sar}{Sardarid\il{NENA!Sardarid}\xspace}
\DeclareRobustCommand{\Sol}{J. Solduz and Šĭno\il{NENA!Solduz and Šĭno (Jewish)}\xspace} % J. Solduz and Šĭno

 
\DeclareRobustCommand{\WNA}{Western Neo-Aramaic\il{Western Neo-Aramaic}\xspace}
\DeclareRobustCommand{\Mal}{Maʿlūla\il{Western Neo-Aramaic!Maʿlūla}\xspace}

\DeclareRobustCommand{\Syr}{Syriac\il{Syriac}\xspace}

\DeclareRobustCommand{\CAram}{Classical Aramaic\il{Aramaic!Classical}\xspace}

\DeclareRobustCommand{\Per}{Persian\il{Persian}\xspace}
\newcommand{\Azr}{Azeri\il{Azeri}\xspace}

% Kurdish dialects

\DeclareRobustCommand{\Sor}{Sorani\il{Kurdish!Sorani}\xspace}
\DeclareRobustCommand{\Kur}{Kurmanji\il{Kurdish!Kurmanji}\xspace}


\DeclareRobustCommand{\KSul}{Silêmanî\il{Kurdish!Silêmanî (Sulemaniyya)}\xspace}
\DeclareRobustCommand{\War}{Wârmâwa\il{Kurdish!Wârmâwa}\xspace}

\newcommand{\Ak}{Akre\il{Kurdish!Akre}\xspace}
\newcommand{\Bin}{Bingird\il{Kurdish!Bingrid}\xspace}
\newcommand{\Muk}{Mukrî\il{Kurdish!Mukrî}\xspace}
\newcommand{\Piz}{Piždar\il{Kurdish!Piždar}\xspace}
\newcommand{\Rdz}{Rewandiz\il{Kurdish!Rewandiz (Ruwandiz)}\xspace}
\newcommand{\Am}{Amêdî\il{Kurdish!Amêdî (Amədya)}\xspace} % Kurdish Amediya
\newcommand{\Sin}{Sinǰar\il{Kurdish!Sinǰar}\xspace}
\newcommand{\Diar}{Diyarbakır\il{Kurdish!Diyarbakır}\xspace}
\newcommand{\Sur}{Sûrçi\il{Kurdish!Sûrçi}\xspace}
\newcommand{\Yer}{Yerevan\il{Kurdish!Yerevan}\xspace}

% Gorani
\DeclareRobustCommand{\Hawr}{Hawrami\il{Hawrami}\xspace}
\newcommand{\Gawr}{Gawraǰū\il{Gawraǰū}\xspace}


\DeclareRobustCommand{\JBA}{Jewish Babylonian Aramaic\il{Aramaic!Jewish Babylonian}\xspace}
\DeclareRobustCommand{\CMand}{Classical Mandaic\il{Mandaic!Classical}\xspace}
\DeclareRobustCommand{\NMand}{Neo-Mandaic\il{Mandaic!Modern (Neo-Mandaic)}\xspace}

\DeclareRobustCommand{\Midn}{Mīdin\il{NWNA!Mīdin} \xspace}

\DeclareRobustCommand{\Arab}{Arabic\il{Arabic!Standard}\xspace} % Standard Arabic
\DeclareRobustCommand{\Malt}{Maltese\il{Maltese}\xspace}
\DeclareRobustCommand{\Morc}{Moroccan Arabic\il{Arabic!Moroccan}\xspace}
\DeclareRobustCommand{\Iraq}{Iraqi Arabic\il{Arabic!Iraqi}\xspace}
\DeclareRobustCommand{\CArab}{Classical Arabic\il{Arabic!Classical}\xspace}

\DeclareRobustCommand{\BHeb}{Biblical Hebrew\il{Hebrew!Biblical}\xspace}
\DeclareRobustCommand{\MishHeb}{Mishnaic Hebrew\il{Hebrew!Mishnaic}\xspace}
\DeclareRobustCommand{\MHeb}{Modern Hebrew\il{Hebrew!Modern}\xspace}

\DeclareRobustCommand{\Akk}{Akkadian\il{Akkadian}\xspace}

\DeclareRobustCommand{\Turk}{Turkish\il{Turkish}\xspace}

\makeatother


\renewcommand{\Ref}[1]{\ref{#1}}
\newcommand{\sref}[1]{\ref{#1}}
\newcommand{\Sref}[1]{{Section \ref{#1}}}
\newcommand{\cref}[1]{\ref{#1}}
\newcommand{\Cref}[1]{{\ref{#1}}}
\renewcommand{\Vref}[1]{{\ref{#1}}}

\renewcommand{\N}{\textbf{N} } % for Mutzafi's references
\renewcommand{\Q}{\textsc{q}} % in table of diachrony2

\NewBibliographyString{fromNENA}
\NewBibliographyString{fromnena}

\DefineBibliographyStrings{english}{%
  bibliography = {References},
  nodate = {{}n.d.},
  fromgerman = {from German},
  fromNENA = {from Neo-Aramaic},
  fromnena = {from Neo-Aramaic},
}


% The following will get rid of some error messages related to linguex.
\makeatletter
\DeclareOldFontCommand{\rm}{\normalfont\rmfamily}{\mathrm}
\makeatother


\newcommand{\name}[2]{#1 #2\ia{#2, #1@#2, #1}\xspace}

\let\oldtoprule\toprule
\let\oldbottomrule\bottomrule
\renewcommand{\toprule}{\midrule\oldtoprule}
\renewcommand{\bottomrule}{\oldbottomrule\midrule} 

\newcommand{\antipar}{\vspace*{-2mm}}

\csname @Latintrue\endcsname

\renewcommand{\cite}{\citealt}