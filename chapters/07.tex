






\chapter{Attributive constructions in the Jewish dialect of Urmi}

\renewcommand{\defaultDialect}{\JUrm}

The data for \JUrm is based on two different sources: \citet{GarbellUrmi}\footnote{Garbell's description covers several dialects of Iranian Azerbaijan as well as the neighbouring \ili{Turkish} territory, which she divides into a northern group (\JUrm and neighbouring dialects) and a southern group (\Sol). The data used here is based on her description of the northern group, of which \JUrm is considered to be representative. \label{ft:two_sources}} and \citet{KhanUrmi}.\footnote{I have also conducted fieldwork with Rabbi Ḥaim Yeshurun, a speaker of \JUrm currently living in Israel, and have consulted the corpus of \citet{BenRahamim}. Though no examples from these sources are presented here, both Yeshurun and \citeauthor{BenRahamim}'s texts are in accordance, according to my examination, with the descriptions of \citeauthor{GarbellUrmi} and \citeauthor{KhanUrmi}, especially in what regards the AC system.} For consistency, I use the transcription system of \citeauthor{KhanUrmi}.\footnote{Note especially the marking of velarized words by an initial ⁺ sign. In some examples I have added a missing velarization mark between parenthesis \parplus, when such a mark was justified according to the sources' lexica. Note that in Garbell's work, the lack of velarization may be due to dialectal variation, as her work describes several related dialects, as explained in \vref{ft:two_sources}.} In citations of examples from \citeauthor{KhanUrmi}, the reference to the section number in the corpus is given in square brackets, when available. 

The richest and most prominent \isi{attributive construction} strategy of \JUrm is head-marking. Pronominal head marking is covered in \ref{ss:JUrmi_Poss_Pro} while the \isi{construct state} construction is addressed at \ref{ss:JUrmi_CST}. There are reasons to believe that in \JUrm, unlike in  \JZax and \Qar, the \cst* \ed suffix is a word-level inflectional suffix rather than a \isi{phrasal suffix}. 

Alongside the CSC one finds in \JUrm the \isi{analytic linker construction} using an alternative \lnk* form \transc{ay} (see \sref{ss:JUrm_Lnk}). Moreover, it differs from the typical ALC in that the \prim is normally marked as \cst*, making it a double-marked construction. Another additional marking is the usage of  \gen* marking, covered in \sref{ss:JUrm_gen}, yielding triple-marked constructions. The classical \isi{double annexation} construction, occurring rarely in \JUrm, is covered in \sref{ss:JUrm_double}. 

The usage of the borrowed \rel* \transc{ki} before clausal \secns is discussed in \sref{ss:JUrm_Rel}. Finally, the  use of \isi{juxtaposition} is presented in \sref{ss:JUrm_juxt}. The usage of {inverse juxtaposition}\isi{inverse juxtaposition construction} (where the \secn precedes the \prim) is presented in \sref{ss:JUrm_inverse}, together with the claim that it arose due to \isi{language contact}.




\section{Possessive pronominal suffixes (X-y.\poss)} \label{ss:JUrmi_Poss_Pro}

In \JUrm, as in other North-Eastern Neo-Aramaic dialects, a possessive \isi{pronominal suffix} can attach directly to a noun or to a preposition:

\acex{Noun}{Pronoun}{101}
{bel-ew}
{house-\poss.3\masc}
{his house}
{KhanUrmi}{56}

\acex{Preposition}{Pronoun}{190}
{mənn-ew}
{from-\poss.3\masc}
{from him}
{KhanUrmi}{196} 

Two possessed nouns can sometimes be conjoined asyndetically, with the repetition of the \isi{possessive pronoun}:

\acex
{Asyndetically Conjoined Nouns}{Pronoun}{1922}
{da-ew dad-ew}
{mother-\poss.3\masc{} father-\poss.3\masc}
{his parents}
{Garbell1965impact}{171}

\citet[171, \S 2.32.11]{Garbell1965impact} attributes the availability of this constrution to \Azr influence: \blockquote{On the border between syntactical and stylistical interference of T[urkish \ili{Azeri}] with the dialect is the extremely frequent occurrence of two asyndetic heads in a nominal phrase, more often than not alliterative}. She compares the above example to the following \Azr example:\footnote{The asyndetic construction is not limited to possessed nouns. \citet[171]{Garbell1965impact} gives also the \Azr example \foreign{gille glale}{grasses strings, vegetation}.}

\acex[\Azr]
{Asyndetically Conjoined Nouns}{Pronoun}{Azeri_conj}
{anna-si baba-si}
{mother-\poss.3\sg{} father-\poss.3\sg}
{his parents}
{Garbell1965impact}{171}






\section{The construct state construction (X.\textsc{cst} Y)} \label{ss:JUrmi_CST}
\subsection{Introduction}
\JUrm, like the \ili{NENA} dialects surveyed in the previous chapters, has a suffixed \cst* marker \ed, related to the \il{Aramaic!Classical}Classical Aramaic \lnk* \d. \citet[171]{Garbell1965impact} claims this \textquote{is clearly due to the impact of the K[urdish] relation suffix \transc{-i} [=\ez*]}, an idea which is discussed in \sref{ss:role_contact}.   The exact form of the suffix varies: Normally it is realised as \phonemic{əd}\~\phonemic{ət}, but following stems whose last vowel is \phonemic{a} it is optionally realised as \phonemic{at}\~\phonemic{ad} as in \example{111} \citep[174]{KhanUrmi}, a phenomenon which may be attributed to local vowel harmony.\footnote{In this it may reflect some influence of \ili{Turkish}, in which vowel harmony is abundant.}

\acex{Noun}{Noun}{102}
{tar-əd bela}
{door-\cst{} house}
{the door of the house}
{KhanUrmi}{174}

\acex{Noun}{Noun}{111}
{⁺dá-ət/-at brona} 
{mother-\cst{} son}
{the mother of the boy}
{KhanUrmi}{174}

\acex{Noun}{Noun}{135}
{gor-ət tre reše}
{man-\cst{} two heads}
{the man of two heads}
{GarbellUrmi}{86}

In contrast to \JZax (\example{338}) and \Qar (\example{535}), I could not find in my survey of \JUrm  cases where an NP is marked phrase-finally by a single \ed suffix. Such cases seem to require the ALC (see \sref{ss:JUrm_Lnk}). An apparent exception is \example{139}, where a \isi{participial} phrase is marked by a final \ed suffix. Yet this apparent phrasal-final marking is only possible since the participle itself is the last element of the phrase.\footnote{In this I disagree with \citet[230]{KhanUrmi}, who draws from this example the quite general conclusion that \enquote{[i]f the head of the \isi{annexation} [the \prim] consists of a phrase in which one noun is dependent on another, the \isi{annexation} inflection is placed only on the head of this phrase}.} Another possible deviation is shown in \example{158}. There an optional \cst* suffix appears phrase-finally on a noun itself serving as \secn of a CSC (thus not being the head of the NP). This may be explained as a product of a wrong bracketing of the expression, in analogy to an [x\ed [y\ed z]] expression (instead of the actual [[x\ed y] z] required by the semantics).   

Notwithstanding these exceptional cases, in \JUrm it seems safe to analyse \ed as a word-level inflectional marker, rather than a \isi{phrasal suffix}.\footnote{Cf.\ \citet[54, \S 2.12.2]{GarbellUrmi}, who treats the \cst* markers, both the \ed suffix and \isi{apocope}, as \enquote{inflection in relation} of nouns.} Such an analysis is corroborated by cases where two asyndetically conjoined nouns occur in the \prim slot, and each is marked by an \ed suffix (\ref{ex:136}=\example{136bis}):

\acex{Asyndetically Conjoined Nouns}{Noun}{136}
{[id-əd reš-əd] gor-aw}
{hand-\cst{} head-\cst{} man-\poss.3\fem}
{the hands and head of her husband}
{GarbellUrmi}{86}

\acex
{Asyndetically Conjoined Nouns}{Noun}{1923}
{[naš-ət xəzmaw-ət] ⁺hatān}
{people-\cst{} relatives-\cst{} groom}
{the family and relations of the bridegroom}
{GarbellUrmi}{86}

From a \isi{language contact} angle, one may relate such cases to the availability of \isi{asyndetic conjunction} in \JUrm,  which \citet[171]{Garbell1965impact} attributes to \Azr influence (see discussion of \example{1922}). While the \isi{asyndetic conjunction} of nouns is not restricted to ACs, the repetition of the \ed suffix may facilitate the \isi{asyndetic conjunction}, as the conjoined nouns are often alliterative in this construction. Note, moreover, that the \ed suffix seems to block the occurrence a coordinating conjunction following it.\footnote{Judging by the few examples I have, the conjoined possessed nouns appear to be semantically inalienable nouns, but it is not clear whether this a real restriction of the construction.}

\largerpage
One finds \isi{asyndetic conjunction} of nouns  also in \secn position, as in the following example. \citet[235]{KhanUrmi} restricts the occurrence of \isi{asyndetic conjunction} (whether in an AC or not) mainly to a \enquote{few sets of tightly-knit nouns}.

\acex{Noun}{Asyndetically Conjoined Nouns}{221}
{šúl-ət [góra baxtá]}
{affair-\cst{} man woman}
{the affairs of a husband and wife}
{KhanUrmi}{230 {[48]}}



 
\clearpage 
\subsection{Apocopate construct state marking} 
\largerpage
Alongside the suffixed \cst* \ed morpheme, \JUrm can mark \cst* nouns by means of \isi{apocope}. In these nouns, the final \free*  vowel, typical of words of native Aramaic origin, is elided (see \sref{ss:apcopate} for a discussion of the development of these forms). 

\acex{Noun}{Noun}{125}
{bron əčči šənne}
{son.\cst{} sixty years}
{a man 60 years old}
{GarbellUrmi}{86}	


In feminine nouns, except \foreign{brata}{daughter}, the \fem*-gender marker \transc{-t-}, is elided as well \citep[55]{GarbellUrmi}, unlike in the apocopate construct of \il{Aramaic!Classical}Classical Aramaic. Contrast the following two examples:

\acex{Noun}{Noun}{156}
{pqar ay d-o gora}
{neck.\cst{} \lnk{} \gen-\dem.\far.\masc{} man}
{the neck of that man}
{GarbellUrmi}{87} 

\acex{Noun}{Noun}{109}
{brā́t ⁺šultanà}
{daughter.\cst{} king}
{the daughter of the king}
{KhanUrmi}{175 {[29]}}


\citet[175]{KhanUrmi} qualifies the occurrence of the apocopate CSC as happening \enquote{occasionally}, and gives no semantic or functional qualifications of it. \citet[55]{GarbellUrmi} sees the \isi{apocope} as a zero suffix being in free variation with the \ed suffix (the only restriction being that the stem should not end in a consonant cluster). Indeed, judging by the examples, the two types of marking are functionally equivalent, though the suffixed marking seems to be more frequent.

Prosodically, when the \prim is marked by \isi{apocope}, it is sometimes devoid of stress, and cliticizes to the \secn, as in the following example: 

\acex{Noun}{Noun}{110}
{tār\cb{} šəmmé}
{door.\cst\cb{} heavens}
{the door of heaven}
{KhanUrmi}{175 {[52]}}

\newpage 
Some nouns have an apocopate form which is not restricted to the contexts where one would expect a \isi{construct state} noun (i.e., head of a CSC). \citet[161]{KhanUrmi} lists the nouns \foreign{naša}{person} and \foreign{gaba}{side} as having short variants \transc{naš} and \transc{gab} respectively, used \enquote{predominantly when they are indefinite}. Diachronically, these apocopate forms may be derived from the \isi{absolute state}, which was used in indefinite contexts (see \sref{ss:state}).
 As it is hard to establish whether such forms should be analysed as marked for \cst* when they serve as \prims, I have in general not analysed them as being marked for \cst*.

\subsection{Adjectival \prims} 

An adjective modified by a noun can appear as a \prim of the CSC:

\acex
{Adjective}{Noun}{1928}
{ó ⁺torbá ⁺mlit-ə́t fəssé dehwéˈ}
{\dem.\far{} bag full.\fem-\cst{} money gold\_piece.\pl{}}
{the bag full of gold coins}
{KhanUrmi}{219 {[60]}}


An adjective appearing in the \prim position of the CSC followed by a \pl* noun yields a superlative meaning. The grammatical information regarding the referent (gender and number) is marked inflectionally on the adjective, which is the syntactic head. A similar construction is found \JZax (see \example{335}).


\acex{Adjective}{Noun}{112}
{sqəl-t-ət niše}
{beautiful-\fem-\cst{} women(\pl)}
{The most beautiful woman}
{GarbellUrmi}{55}



\subsection{Adverbial \prims}

Some prepositions may be marked by the \cst* suffix when they are complemented  by a noun. Mostly this is optional (see  \sref{ss:JUrm_AdverbialHeads} for examples), 
 but for some prepositions it seems to be obligatory. Thus, the preposition \foreign{bod}{because}, adapted from the Kurdish preposition \foreign{bo}{because}, always appears with a \cst* suffix \transc{-d} \citep[166]{Garbell1965impact}.

\acex{Preposition}{Noun}{193}
{gá-at Urmì}
{in-\cst{} U.}
{in Urmi}
{KhanUrmi}{194 {[136]}}


\acex{Preposition}{Noun}{191}
{mənn-ət bela}
{from-\cst{} house}
{from the house}
{KhanUrmi}{196}

\acex{Preposition}{Asyndetically Conjoined Nouns}{202}
{bá-at [⁺kalo ⁺hatā̀n]}
{for-\cst{} bride groom}
{to the bride and groom}
{KhanUrmi}{193 {[93]}}


\acex{Preposition}{Pronoun}{266}
{bo-d\cb{} ma}
{because-\cst\cb{} what}
{Why?}
{KhanUrmi}{191}

Similarly, some conjunctions are in fact interrogative adverbs augmented with the \cst* suffix (cf.\ \examples{247}{248}).

\acex{Conjunction}{Clause}{259}
{kəmm-ə̀t ⁺məss-étun}
{how\_much-\cst{} can-2\pl}
{as much as you can}
{KhanUrmi}{371 {[67]}}

\acex{Conjunction}{Clause}{260}
{imắn-ət àd-e}
{when-\cst{} \sbjv\footnotemark.come-3\masc}
{whenever he comes}
{KhanUrmi}{372}

\footnotetext{The \sbjv{} is marked by the lack of an indicative prefix \transc{ad-} attached to the stem.}

\subsection{The pronominal \prim \transc{od}} \label{ss:JUrm_od}

In rare cases, the distal \sg* \dem* \transc{o} appears in the \cst* form \transc{od} after some prepositions, such as \foreign{bod}{because} (itself marked obligatorily by the \cst* suffix; see \example{266}).\footnote{In this I agree with \citet[374]{KhanUrmi}, but disagree with \citet[61]{GarbellUrmi} who sees the \ph/od/ segment as an allomorph of the \cst* ending.} This pronoun allows the introduction of a clausal complement of the preposition (yielding an apparent composite conjunction \transc{bod od}).
 
\acex{Pronoun}{Clause}{263}
{bo-d\cb{} ó-d hála zùrt \cb{}ela}
{because-\cst\cb{} \dem-\cst{} still young \cb{}\cop.\fem}
{because of the fact that she is still young}
{KhanUrmi}{374 {[74]}}

The \cst* pronoun \transc{od} is compatible with the \isi{relativizer}, as is shown in \examples{264}{265}.



\subsection{Pronominal, ordinal and adverbial \secns}

A pronoun can occupy the \secn slot. The following two examples present both the use of an indefinite (interrogative) pronoun and a definite (reflexive) pronoun in this position:

\acex{Noun}{Pronoun}{128}
{bel-ət máni}
{house-\cst{} who}
{whose house}
{GarbellUrmi}{86}

\acex{Noun}{Pronoun}{284}
{(g\cb{}) bel\cb{} nòš-u}
{in\cb{} house.\cst\cb{} \refl-3\pl}
{(in) their own house}
{KhanUrmi}{215 {[158]}}

Modification by ordinal numerals uses the same construction, whether the \isi{numeral} is marked by the ordinal suffix \transc{-minji},\footnote{\citet[166, \S 1.22.3]{Garbell1965impact} explains this suffix is a combination of the \Sor (originally \Per) ordinal suffix \transc{-emîn} and the \Azr ordinal suffix \transc{-ji}.} or not:

\acex
{Noun}{Ordinal}{132}
{baxt-ət awwal}
{woman-\cst{} first}
{the first woman}
{KhanUrmi}{187}

\acex{Noun}{Ordinal}{130}
{yom-ət tre-mənji}
{day-\cst{} two-\ord}
{the second day}
{GarbellUrmi}{86}

\acex{Noun}{Ordinal}{131}
{bel-əd arbi \cb{}w xa}
{house-\cst{} forty \cb{}and one}
{the forty-first house}
{GarbellUrmi}{86}

Adverbial modification (either by a PP or by an adverb) can also occur within this construction. While this is attested also in other dialects (see \vref{ss:JZax_adv_secn} for \JZax examples), it seems to be more widespread in \JUrm:

\acex{Noun}{\PP}{148}
{ktab-ət b\cb{} id-ew}
{book-\cst{} in\cb{} hand-\poss.3\masc}
{the book in his hands}
{GarbellUrmi}{87}

\acex{Noun}{Adverb}{283}
{⁺qətt-ət təxya}
{piece-\cst{} below}
{lower piece}
{KhanUrmi}{600}

\acex{Participle}{Adverb}{122}
{⁺samx-an-ət tə́xya}
{stand-\ptcp-\cst{} below}
{the one/those standing below}
{GarbellUrmi}{84}

Note that the last example has a participle as its \prim. \citet[78]{KhanUrmi} notes that normally the participle is used \enquote{as a noun or adjective describing a characteristic, time-stable property of a referent}. This nominal character is retained also when \isi{participles} are complemented by an object, as in the following example, which has a \isi{participial} phrase as its \prim (regarding the initial position of the complement, see \sref{ss:JUrm_inverse_Verbal}).

\acex{Participial Phrase}{Noun}{139}
{[ixala bašl-an-ət] ⁺sultana}
{food cook-\ptcp-\cst{} king}
{the king's cook (lit. \enquote{food cooker})}
{GarbellUrmi}{86}

\subsection{Clausal \secns}

Nouns in \cst* may be followed by \isi{participial} or infinitival phrases, which can be seen as reduced relative clauses (for the order of elements inside these phrases, see  \sref{ss:JUrm_inverse}):

\acex{Noun}{Participial phrase}{147}
{naš-ət [bar-ew yarq-an-e]}
{people-\cst{} after-\poss.3\masc{} run-\ptcp-\pl}
{people that run after him}
{GarbellUrmi}{87}


\acex{Noun}{Infinitival Phrase}{146}
{bel-ət [ixala bašole]}
{house-\cst{} food cook.\inf}
{house of food cooking, kitchen}
{GarbellUrmi}{86}

A noun may also be complemented like this by a clause, but, according to \citet[88]{GarbellUrmi}, only when the clause has no NP functioning as a subject argument (i.e., it has only the obligatory pronominal subject marking on the verb).\footnote{Garbell qualifies this type of clause as a VP, which implies that the subject suffix on the verb is merely an agreement marker. Note, however, that the subject of the embedded clause may be different than the \prim noun, as in example \vref{ex:168}. The only restriction is that the subject must be expressed pronominally as a verbal suffix. See, however, \example{273} for a possible counter-example of Garbell's assertion.} 

\acex{Noun}{Clause}{166}
{šat-ət adya}
{year-\cst{} \sbjv.come.3\fem}
{the coming year}
{GarbellUrmi}{88}

\acex{Noun}{Clause}{168}
{gor-ət [bron-ew ⁺qtə́l-wa-le]}
{man-\cst{} son-\poss.3\masc{} killed-\pst-\agent3\masc}
{the man whose son he had killed}
{GarbellUrmi}{88}

Note that in the last example, \foreign{bronew}{his son} cannot be the subject of the clause due to the above-mentioned restriction on appearance of subject NPs in \secn clauses.

The interrogative pronouns \foreign{ma}{what} and \foreign{măni}{who} can also be complemented by a clause in this construction:

\acex{Pronoun}{Clause}{247}
{má-t abyát}
{what-\cst{} \sbjv.want.2\fem}
{whatever you want}
{KhanUrmi}{358 {[10]}}

\acex{Pronoun}{Clause}{248}
{mắni-t áde}
{who-\cst{} \sbjv.come.3\masc}
{whoever comes}
{KhanUrmi}{357 {[32]}}

Exhibiting such a \prim, the following example shows that Garbell's above-mentioned restriction does not hold. This may be due to the fact that the \cst* suffix has been grammaticalised into the \prim pronoun, and is not felt any more as such. As the counter-example comes from Khan's description (about 40 years after Garbell's) it may reflect, moreover, a subtle \isi{language change}.

\acex
{Pronoun}{Clause}{273}
{má-t nā́š m\cb{} əlhá abèˈ}
{what-\cst{} man from\cb{} God \sbjv.want.3\masc}
{whatever a person wants from God}
{KhanUrmi}{358 {[109]}}



For the alternative strategy of using a \isi{relativizer}, see  \sref{ss:JUrm_Rel}.



\section{The analytic linker construction (X \textsc{lnk} Y)} \label{ss:JUrm_Lnk}
\subsection{Introduction}
\JUrm uses the morpheme \transc{ay}  as a \lnk*, and not the inherited \d as do \JZax and \Qar. It is identical in form to the singular proximal demonstrative \transc{ay}  (which \cite[58]{GarbellUrmi} qualifies as an \enquote{archaic} form), and very probably related to it diachronically, but, contrary to the latter, it does not inflect according to number. Nonetheless, in some cases it is difficult to decide between the two possible analyses (as example \vref{ex:288} shows). 

\citet[171]{Garbell1965impact} relates the \lnk* \transc{ay} to the \textquote{relational morpheme  of [Kurdish, which] is likewise demonstrative in its origin}. She notes moreover that in the related dialects of Southern \ili{Persian} Azerbaijan, \Sol, the \Sor  marker \transc{i} is used in the same position. A similar suggestion is made by \citet[176]{KhanUrmi}: \blockquote{It is likely to have developed under the influence of the \textit{izafe} construction in Iranian languages. It appears not to be a direct loan from Iranian, in which the \textit{izafe} is in principle monosyllabic (\textit{e, i, a}), but rather an imitation of the \textit{izafe} using Aramaic morphological material}. For an evaluation of these proposals see \sref{ss:JUrm_ay}. \il{Iranic} \is{Ezafe}

In general \transc{ay} is an independent phonological word, as it carries stress. Quite often, however, it is found cliticized forward with the \secn or, sporadically, backward with the \prim.\footnote{Information about stress and cliticization is given only in \citet{KhanUrmi}, where \isi{clitic} boundaries are marked by a hyphen.} The latter possibility is especially frequent with \isi{adverbial} \prims, which tend to cliticize forward on their own account (but see \example{219}). 

The pronominal origin of \transc{ay}, and -- more importantly -- the fact that it can form an AC without an explicit \prim (see \sref{ss:JUrm_lnk_noPrim}), are the main motivations to analyse it as a pronominal \lnk*, rather than  a simple \secn marker. As such, it forms an independent syntactic (and sometimes prosodic) constituent with the \secn. From a comparative perspective, it may be seen as the functional equivalent of the \d or \transc{did} \lnk*s of other dialects. Yet  in contrast to the dialects surveyed in the previous chapters, in \JUrm, the \lnk* regularly occurs with \cst-marked \prims, as examined in \sref{ss:JUrm_cst_lnk}. In fact, cases where the \prim is marked by the \free* (i.e., the default form) are only found 
sporadically according to \citet[175]{KhanUrmi}. 







 



\acex{Noun}{Noun}{149}
{o aġa ay ašqalon}
{\dem.\far.\sg{} lord \lnk{} A.}
{that lord of Ascalon}
{GarbellUrmi}{87}

\acex{Noun}{Noun}{150}
{gora ay tre reše}
{man \lnk{} two heads}
{the man of two heads}
{GarbellUrmi}{87}

Yet, as noted in \sref{ss:JUrmi_CST}, whenever the \prim is a noun phrase, rather than a simple noun, the CSC is generally not available, and the ALC becomes the sole option (ignoring possible circumlocutions). Since the \cst* marking in \JUrm is not phrasal, the \prim NP cannot be marked as \cst*: 

\acex{Noun Phrase}{Noun}{157}
{[tq-ət aqla] ay naš}
{place-\cst{} foot \lnk{} man}
{human footprints (lit. place of foot of man)}
{GarbellUrmi}{87}

\acex{Noun Phrase}{Noun Phrase}{179}
{[zóra broná] áy [tmánya ⁺əčča šənnè]}
{small son \lnk{} eight nine years}
{the young boy of eight or nine years}
{KhanUrmi}{175 {[141]}}

In such cases the \lnk* may form an independent prosodic constituent with the \secn:

\acex
{Noun Phrase}{Noun}{180}
{[xa\cb{} danká ⁺torbà]ˈ ay\cb{} ixalàˈ}
{one\cb{} unit bag \lnk\cb{} food}
{a bag of food}
{KhanUrmi}{176 {[22]}}

\acex
{Noun Phrase}{Noun}{181}
{kúl-lu ⁺ktabèˈ ay\cb{} dunyèˈ}
{all-3\pl{} books \lnk\cb{} world}
{all the books of the world}
{KhanUrmi}{176 {[29]}}

\newpage 
There is one case where two separate \secns appear, each in its own prosodic phrase. Note that the \lnk* phrase \foreign{áy awuršúm}{of silk} is adjacent to the adjective \foreign{sqilè}{beautiful.\pl} which refers semantically to the \prim \foreign{gory-ə́t awuršùm}{silk stockings}. Yet syntactically the adjective can be analysed as being an attribute of the pronominal \lnk* itself:

\acex
{Noun Phrase}{Asyndetically Conjoined Nouns}{1929}
{xá\cb{} zoa gory-ə́t awuršùmˈ áy šušà,ˈ áy awuršúm sqil-èˈ goryé mabruq-é mditàˈ bá-at ⁺kalò.ˈ}
{one\cb{} pair stockings.\cst{} silk \lnk{} nylon \lnk{} silk beautiful-\pl{} stockings shining-\pl{} brought.\resl.\fem{} for-\cst{} bride}
{She has brought a beautiful pair of silk stockings, of nylon, of silk, shining stockings for the bride}
{KhanUrmi}{219 {[94]}}

There may also be rare cases where the \lnk* forms a prosodic constituent with the \prim. Yet, such cases may be analysed differently. For instance, in following example the \prim and the \secn are co-referential, atypically for an AC. In this case, the \transc{ay} element may equally well be analysed as a \isi{demonstrative pronoun}, followed by a prosodic stress due to hesitation:\footnote{According to Khan's transcription it is the noun that loses the stress and apparently \enquote{procliticizes} to the stressed \lnk*. Yet, since in \JUrm the default stress placement is on the ultimate syllable, especially before intonation boundaries \citep[46]{KhanUrmi}, it is more plausible to analyse this case as an \isi{encliticization} of the \lnk* to the \prim, with a default stress placement on the resulting phonological word.}  

\acex{Noun}{Noun}{219}
{brona \cb{}àyˈ ⁺hatā́n}
{son \cb{}\lnk/\dem.\sg{} groom}
{the son (who is) the groom}
{KhanUrmi}{219 {[79]}}





\subsection{Linker following a construct state (X-\textsc{cst} \textsc{lnk} Y)} \label{ss:JUrm_cst_lnk}

The typical usage of the \lnk* is following a \cst-marked \prim:

\acex{Noun}{Noun}{152}
{o aġá-ad ay ašqalon}
{\dem.\far.\sg{} lord-\cst{} \lnk{} A.}
{that lord of Ascalon}
{GarbellUrmi}{87}

\acex{Noun}{Noun}{153}
{gor-ət ay tre reše}
{man-\cst{} \lnk{} two heads}
{the man of two heads}
{GarbellUrmi}{87}

This construction can easily be iterated: 

\acex{Noun}{Noun Phrase}{224}
{ó raís-ət áy [komsér-ət áy Urmì]}
{\dem.\far.\sg{} head-\cst{} \lnk{} police-\cst{} \lnk{} U.}
{the head of police of Urmi}
{KhanUrmi}{230 {[134]}}

Note that the \prim can also be marked by the apocopated \cst*, as in the following example (=\example{156}; cf.\ \foreign{pqarta}{neck} in \free*):\footnote{Regarding the \gen* marking, see \sref{ss:JUrm_lnk_gen}.}

\acex{Noun}{Noun}{156bis}
{pqar ay d-o gora}
{neck.\cst{} \lnk{} \gen-\dem.\far.\masc{} man}
{the neck of that man}
{GarbellUrmi}{87}

These cases, appearing regularly, pose a problem to the  analysis of \transc{ay} as a classical pronominal \lnk*. As outlined in \sref{ss:Analytic_AC}, the essence of the classical \ili{Semitic} ALC, is that the \lnk* stands in \isi{apposition} with a nominal in \free*, being outside the scope of the \isi{attributive relation}, strictly speaking. Yet this analysis is not tenable with regard to those cases in which the \prim is explicitly marked by the \cst*. One possible solution is to argue that the \lnk* and the \prim are still in \isi{apposition}, yet the \prim shows \concept{agreement in state} with the \lnk*, which is syntactically in \cst*. 
This means that both the \prim and \transc{ay} head this type of construction, similarly to cases where two \cst* nouns head a CSC, as in \examples{136}{1923}. The resulting construction, moreover, is one cohesive NP, as is clear from the fact that the \prim in \cst* cannot be separated from the following \lnk*. 

 Yet  this analysis is challenged by cases where the \lnk* intervenes between a \cst* marked preposition and its complement: 


\acex{Preposition}{Noun Phrase}{207}
{(zə́l-lu) géb-əd ay\cb{} [⁺rəww-ət ay\cb{} komsèr]}
{went-3\pl{} at-\cst{} \lnk\cb{} chief-\cst{} \lnk\cb{} police}
{They went to the chief of police}
{KhanUrmi}{198 {[127]}}

\acex{Preposition}{Noun}{208}
{gá-at ay\cb{} daxlà}
{in-\cst{} \lnk\cb{} agriculture}
{in agriculture}
{KhanUrmi}{198 {[152]}}

It is not possible to argue that a pronominal element stands in \isi{apposition} with a preposition, as the latter is not nominal. Thus, at least in the  latter cases, \transc{ay} must have lost its pronominal force (i.e.\ the necessity of representing a noun), and has become a simple marker of the \isi{prepositional phrase}, a \concept{pure linker}. As such, it approaches the status of a phrasal \gen* marker. 

\subsection{Syllabification of the construct state suffix with the linker} \label{ss:JUrm_resyl}

Some analytic confusion arises from cases in which the linker \transc{ay} is preceded by \ph/d-/ segment. While it is tempting to simply analyse this segment as the \isi{genitive prefix}, which can occur before vowel-initial pronominal elements (see discussion in \sref{ss:JUrm_gen}), this analysis is  inconsistent with the view advocated here that the linker is not governed by the \prim, but rather stands in \isi{apposition} with it.\footnote{In other words, while the \lnk* marks the \secn, it stands outside it. This is true even if one sees \transc{ay} as a pure marker of the AC, as suggested in the previous section. Of course, this is only the case if the particle under consideration is not in fact  the homophonous \isi{demonstrative pronoun} \transc{ay} in \isi{genitive case}.} 

The solution to this difficulty is to analyse the \ph/d-/ segment as part of the \cst* suffix \ed, which has been resyllabified with the \lnk* due to phonological reasons, in particular the fact that the linker is vowel-initial.\footnote{As \citet[175]{KhanUrmi} in fact suggests: \enquote{The consonant of the genitive \isi{enclitic} [-d] may be syllabified with the \transc{ay} particle.}}


 In some cases, a vestige of the vocalic nucleus of the \cst* suffix (\ph/ə/ or \ph/a/, glossed in both cases \\isi{schwa}) remains attached to the \prim, while in other cases the vowel is elided. 
 







\acex{Noun}{Noun}{175}
{bél-ə -d\cb{} áy ⁺flankás}
{house-\\isi{schwa}{} -\cst\cb{} \lnk{} so\_and\_so}
{the familly of so-and-so}
{KhanUrmi}{175 {[72]}}

\acex{Noun}{Proper Noun}{256part}
{yá ⁺šultán-a -d\cb{} áy {Pahlawi}̀ˈ}
{\dem.\near.\sg{} king-\\isi{schwa}{} -\cst\cb{} \lnk{} P.}
{that king (Reza Shah) Pahlavi}
{KhanUrmi}{370 {[169]}}


A similar \isi{resyllabification}  occurs with prepositional \prims. In these cases it is further motivated by the fact that the prepositions themselves cliticize to the \lnk* as a whole.

\acex{Preposition}{Noun}{209}
{əl-d\cb{} áy ⁺amart-èw}
{\acc-\cst\cb{} \lnk{} palace-\poss.3\masc}
{his palace\footnotemark}
{KhanUrmi}{198 {[45]}} 

\footnotetext{The preposition \foreign{əl}{to} serves in this case as an \acc* preposition, i.e.\ marking an object of a phrase. It is thus not rendered in the translation.}

\acex{Preposition}{Noun}{210}
{ba-d\cb{} áy elčyè}
{for-\cst\cb{} \lnk{} messengers}
{for the messengers}
{KhanUrmi}{198 {[77]}}

Note that in the last examples the \cst* morpheme is realized as a sole \phonemic{d} without a vocalic nucleus: Contrast this example with \example{202}, where it is realized with a vocalic nucleus as \phonemic{-at}. 



\subsection{Adjectival \secns following an apparent linker} \label{ss:JUrm_apparant_adj_secns}

There are sporadic cases where an adjective follows an \transc{ay} morpheme, which \textit{a priori} could be analysed as a \lnk*. Such an analysis would reinforce the idea that the \JUrm ALC is a \isi{pattern replication} from \Kur, where adjectives follow the \ez* (see \sref{ss:kurd_lnk_adj}). Closer scrutiny nevertheless suggests that, given the rarity of such occurrences, the \transc{ay} in these cases is actually the \isi{demonstrative pronoun}. Thus, in the following example, \transc{ay}  can be analysed as a definite determiner attached to the adjective rather than the possessed noun (compare with \JZax \example{462}):\footnote{As discussed in \vref{ft:ex462}, Eran Cohen has suggested  that this very placement of a determiner in the pre-adjectival position may itself represent \isi{pattern replication} of the \ez* construction. If this is true, this development may be regarded as a pre-cursor to the re-analysis of the \dem* as a \lnk*. Yet the rarity of this construction could argue against such a scenario, at least in the context of \JUrm. An alternative possibility is to consider \transc{\+rast} as a noun meaning \transl{the right side}; see discussion of the \Kur \example{725}.}

\acex{Noun}{Adjective}{228}
{[kpan-aw] ay ⁺rast}
{shoulder-\poss.3\fem{} \defi.\sg{} right(\invar)}
{her right shoulder}
{GarbellUrmi}{87}





In the following example \transc{ay} is also a definite determiner, serving to nominalize the adjective:


\acex{Preposition}{Adjective}{213}
{gáll-ə -d\cb{} áy smoqà,ˈ idá smoqàˈ}
{with-\\isi{schwa}{} \cst\cb{} \defi.\sg{} red.\masc{} hand(\masc) red.\masc}
{with red, a red hand}
{KhanUrmi}{198 {[173]}}

\largerpage
\subsection{Infinitival phrases as \secns}

Infinitive phrases as well can appear as \secns in the \JUrm ALC.

\acex{Noun}{Infinitive}{234}
{(gə\cb{}) tkán-ə -d\cb{} áy [ləxmá zabonè]ˈ}
{in shop-\\isi{schwa}{} -\cst\cb{} \lnk{} bread sell.\inf}
{in the shop of bread selling}
{KhanUrmi}{291 {[174]}}

\acex{Noun}{Infinitive}{235part}
{léle -d\cb{} áy [pardìn šaroé]}
{night -\cst\cb{} \lnk{} curtain untie.\inf}
{the night of the releasing of the curtain}
{KhanUrmi}{291 {[84]}}



\clearpage 
\subsection{Ordinal and adverbial \secns}

 Also \isi{ordinals} and adverbs can serve as \secns in the ALC:

\acex{Noun}{Ordinal}{155}
{o gor-ət ay tre-mənji}
{\dem.\far.\masc{} man-\cst{} \lnk{} two-\ord}
{that second man}
{GarbellUrmi}{87}

Compare the above example to \vref{ex:130}.

\acex{Noun}{Adverb}{225}
{⁺qayd-ət ay\cb{} lòkaˈ}
{custom-\cst{} \lnk\cb{}\footnotemark{} there}
{the custom of the place}
{KhanUrmi}{230 {[151]}}

\footnotetext{Arguably, the \transc{ay} element here could be an instance of the demonstrative \transc{ay} serving to nominalize the adverb. Indeed, \citeauthor{KhanUrmi} translates  this example as \transl{the custom of \textit{that} place}. Compare with \example{182} and see also \example{214} and the discussion following it.}

\subsection{Linkers without an explicit \prim} \label{ss:JUrm_lnk_noPrim}

As explained in \sref{ss:Analytic_AC}, the ALC is analysed as a construction in which the linker is standing in an \isi{attributive relationship}  with the \secn and in \isi{apposition}  with the \prim. As such, the linker is expected to be able to occur without an explicit \prim. This expectation is indeed borne out, but only when the linker phrase acts as the predicate of the clause.

\acex{\zero}{Noun}{162}
{ay šabbat}
{\lnk{} Sabbath}
{belonging to the Sabbath}
{GarbellUrmi}{88}

\acex{\zero}{Ordinal}{161}
{ay arbi}
{\lnk{} forty}
{the fortieth}
{GarbellUrmi}{88}

\acex{\zero}{Pronoun}{227}
{(kalòˈ) ay\cb{} noš-èw (\cb{}ila.ˈ)}
{bride \lnk\cb{} \refl-3\masc{} \cb{}\cop.\fem}
{The bride belongs to him.}
{KhanUrmi}{233 {[81]}}

\acex{\zero}{Infinitive}{239}
{áy šaqolè (\cb{}le)}
{\lnk{} buy.\inf{} \cb{}\cop.\masc}
{It is worth buying.}
{KhanUrmi}{292}

\acex{\zero}{Infinitival Phrase}{240}
{áy [ləbbá qyalà] (\cb{}we-la)}
{\lnk{} heart burn.\inf{} \cb{}\cop.\pst-\fem}
{It was liable to burn the heart, it was pitiable.}
{KhanUrmi}{293 {[121]}}

\acex{\zero}{Asyndetically conjoined infinitives}{230}
{bắle fkə́r wad-én ki\cb{} did-ànˈ  ⁺rába ⁺rába \zero{} ay\cb{} xazoèˈ rába \zero{} ay\cb{} šamoè ilá.ˈ}
{but thought do-1\masc{} \rel\cb{}{} \gen-1\pl{} much much \zero{} \lnk\cb{} see.\inf{} much \zero{} \lnk\cb{} hear \cop.3\fem}
{but I think that our (wedding) is very much something to see and something to hear about}
{KhanUrmi}{233 {[71]}}




One case where a \lnk* phrase is found in a non-predicative position is after the \isi{adverbial} \foreign{magon}{like}, itself appearing in predicative position:


\acex{\zero}{Adverb}{214}
{magón \zero{} ay\cb{} láxxa là k-awyá-waˈ}
{like  \zero{} \lnk\cb{} here \neg{} \ind-be.\fem-\pst}
{It was not like (the situation) here}
{KhanUrmi}{198 {[106]}}

In this case, one may reasonably interpret the \transc{ay} as referring pronominally to an implicit situation, in essence nominalizing the adverb \foreign{laxxa}{here}. An alternative analysis would be to see \transc{ay} as a pure \lnk* standing between \transc{magon} and its complement, but in this case one would expect \transc{magon} to be marked with the \cst* suffix, as in \examples{207}{208}. 

\section{Genitive marking of \secns } \label{ss:JUrm_gen}

\subsection{Introduction}

Demonstrative and interrogative pronouns, which are normally vowel-initial,\footnote{\JUrm\ does not have initial glottal stops \citep[35]{KhanUrmi}.} are marked by a \gen* prefix \transc{d-} when they appear as \secns, either as \isi{determiners} of NPs or as full NPs in their own right. The motivation to analyse this segment as a \gen* marker, rather than a phonological artefact, is given in \sref{ss:d_gen}. The possible development path of this marker is discussed in \sref{ss:genitive_development}.



In general, the \prim is expected to be marked as \cst*, either by means of the \ed suffix or by \isi{apocope}. This exception is indeed borne out in some cases:

\acex{Noun}{Noun}{108}
{dád-ət d-ò broná}
{father-\cst{} \gen-\dem.\far.\sg{} son}
{the father of that son}
{KhanUrmi}{175 {[70]}}

\acex{Noun}{Pronoun}{127}
{baxt-əd d-ay}
{wife-\cst{} \gen-\dem.\near.\masc}
{the wife of this (man)}
{GarbellUrmi}{86}


\acex{Noun}{Noun}{288}
{bel d-ay gora}
{house.\cst{} \gen-\dem.\near.\masc{} man}
{the house of this man}
{KhanUrmi}{175}





\citet[175]{KhanUrmi} brings some cases in which the \prim is apparently left unmarked:\footnote{In Khan's analysis this is the default case. For him, there is only one \transc{d} morpheme, which is a particle that can attach either to the \prim as an \enquote{\isi{annexation} enclitic} or to the \secn's determiner as a prefix. Moreover, he does not consider reduced nouns, such as \foreign{bel}{house} in example \vref{ex:288}, to be in \cst*, but rather as lacking the \isi{enclitic}, which is attached to the following demonstrative \citep[174--175]{KhanUrmi}. See \sref{ss:d_vs_ed} for  arguments against such an analysis.} 

\acex{Noun}{Noun}{105}
{ni⁺šā́n d-o pardá}
{sign.(\cst?) \gen-\dem.\far.\sg{} curtain}
{the symbolic meaning of that curtain}
{KhanUrmi}{175 {[88]}}



\acex{Noun}{Noun}{106}
{áy xabúša d-émnu yalè \cb{}le?}
{\dem.\near.\sg{} apple \gen-which child \cb{}cop.3\masc}
{This apple belongs to which child?}
{KhanUrmi}{175}

Both these examples may be explained differently, however: In example \ref{ex:105}, the \prim \foreign{ni\+šā́n}{sign} can be analysed as being an instance of apocopated \cst*, since there exists a long variant \transc{ni\+šā́na}. 

As for the \example{106}, judging by Khan's translation (\transl{This is the apple of which child?}), it seems that he analyses \transc{xabúša d-émnu yalè} as one NP. Yet it seems more reasonable to analyse \transc{áy xabúša} as the subject NP and \transc{d-émnu yalè} as a predicate NP, in which case \transc{d-émnu} is an independent genitive pronoun, lacking an explicit \prim. Functionally, the \isi{genitive marking} without an explicit \prim is somewhat similar to the pronominal \lnk* \d, otherwise absent in \JUrm, as it can be said to assume a pronominal role of representing the \prim.

\subsection{Genitive marking following adverbials}

Prepositions stand in a direct \isi{attributive relation} with their complement and thus induce a \gen* marking on it, irrespective of the question whether the preposition itself is marked as \cst* or not

\acex{Preposition}{Noun}{195}
{dowr-ət d-o bela}
{around-\cst{} \gen-\dem.\far.\sg{} house}
{around that house}
{KhanUrmi}{194}

\acex{Preposition}{Pronoun}{194}
{dowr-ət d-o}
{around-\cst{} \gen-\dem.\far.\sg}
{around that one}
{KhanUrmi}{194}

\acex{Preposition}{Noun}{185}
{bar d-o gora}
{behind \gen-\dem.\far.\sg{} man}
{behind that man}
{KhanUrmi}{192}

\acex{Preposition}{Pronoun}{188}
{bar\cb{} d-o}
{behind\cb{} \gen-\dem.\far.\sg}
{behind that one, behind him}
{KhanUrmi}{192}





Of interest are cases in which the \isi{adverbial} has a part of the \cst* suffix, namely the vowel \ph/ə/, followed by a \gen* marked  pronoun. Such cases are akin to those in which the \ph/-d/ segment of the \cst* suffix has resyllabified with the following element (compare with \sref{ss:JUrm_resyl}).  Given, however, that pronominal \isi{determiners} are normally marked by the \gen*, one must conclude that the \d prefix does double duty in these cases, serving both to the \cst* suffix and to the \gen* prefix. In other words, while phonologically it is a simple \ph/d/ segment, syntactically it is understood as geminated.\footnote{This analysis is independent of the question whether diachronically there was a geminated \ph/d-d/ in this position. In fact, the \isi{resyllabification} of the \cst* \ed suffix may have been the trigger for the innovation of the \d \gen* prefix, prior to any gemination. See discussion in \sref{ss:genitive_development}.}




\acex{Preposition}{Noun}{199}
{⁺g\cb{}aralġ-ə -d-emnu naše}
{in\cb{}between-\\isi{schwa}{} -\cst.\gen-which people}
{between which people?}
{KhanUrmi}{196}

\acex{Preposition}{Noun}{201}
{⁺mqulb-ə\footnotemark{} -d-o gora}
{instead-\\isi{schwa}{} -\cst.\gen-\dem.\far.\sg{} man}
{instead of that man}
{KhanUrmi}{196}

\footnotetext{\citeauthor{KhanUrmi} analyses \transc{⁺mqulb} as consisting of the prefix \foreign{m-}{from} and a Kurdish (originally \ili{Arabic}) element \transc{qulb} \citep[569]{KhanUrmi}.}










\largerpage

\subsection{Independent genitive pronouns} \label{ss:JUrm_ind_gen}

All personal pronouns have a genitive allomorph, which takes the general form of  \transc{did}+\poss\ \citep[58]{KhanUrmi}. These forms appear whenever one expects a pronoun in an attributive position and this pronoun cannot be expressed as a \isi{pronominal suffix} for morphological reasons (\cite[233]{KhanUrmi}; contrast with \examples{101}{190}). 

\acex{Noun}{Pronoun}{229}
{kalo did-ew}
{bride \gen-\poss.3\masc}
{his bride}
{KhanUrmi}{233}

\acex{Preposition}{Pronoun}{187}
{bo-d\cb{} did-ew}
{because-\cst\cb{} \gen-\poss.3\masc}
{because of him}
{KhanUrmi}{192}



The first example has a \prim noun \foreign{kalo}{bride} which does not end in an \phonemic{a} vowel, and thus cannot take a regular \isi{pronominal suffix}. Similarly, the preposition \foreign{bod}{because} cannot take a \isi{pronominal suffix}, since it ends obligatory with the \cst* marker.













Note that semantically there is a certain overlap between the \third\ person genitive pronouns (\masc: \transc{didew}; \fem: \transc{didaw}; \pl: \transc{didu}) and the independent demonstratives with \isi{genitive marking} (\sg: \transc{do}; \pl: \transc{dune}; see examples \ref{ex:194} and \vref{ex:188}). This is expected, as the same overlap appears in the non-\isi{genitive case}. Indeed, without the \isi{genitive marking}, the distal demonstratives \foreign{o}{that} and \foreign{une}{those} are identical to the independent pronouns \citep[55--56]{KhanUrmi}. 

In \JUrm, the \transc{did-} base is bound to the pronominal suffixes, and cannot occur before free standing nominals. Consequently I analyse it as  a genitive base, and not as a pronominal \lnk*, in contrast to \JZax (see \sref{ss:JZax_Lnk}) and \Qar (\examples{523}{521}). This point will be discussed in more detail in \sref{ss:pron_base}. Unlike the \transc{d-} genitive marker, however, in some cases it has a certain pronominal value, especially when it acts as the head of an NP in predicative position (but see \example{106} for a similar analysis of the \transc{d-} prefix). Formally, I analyse such cases as having a \zero\ \prim, as an explicit noun could appear in the \prim position. 

\acex{\zero}{Pronoun}{231}
{ma-t ə́t-ti l-ə́t-ti kúll-u \zero{} did-àx \cb{}ilu.ˈ}
{what-\cst{} exist-1\sg{}  \neg-exist-1\sg{} all-3\masc{} \zero{} \gen-2\fem{} \cb{}\cop.3\masc}
{Whatever I have is all yours.}
{KhanUrmi}{233 {[8]}}



\subsection{Genitive case following the linker \transc{ay}} \label{ss:JUrm_lnk_gen}

A demonstrative or \isi{interrogative pronoun} appearing after the linker will also be marked by the \isi{genitive prefix} \transc{d-}. Together with the \cst* marking, this yields a triple-marked construction (see also \example{156}):

\acex{Noun}{Noun}{154}
{tre be-ət ay d-ay gora}
{two eggs-\cst{} \lnk{} \gen-\dem.\near.\sg{} man}
{the two eggs of this man}
{GarbellUrmi}{87}

\acex{Noun}{Noun}{182}
{⁺qayd-ət áy d-ò\cb{} tka}
	{custom-\cst{} \lnk{} \gen-\dem.\far.\sg\cb{} place}
{the custom of that place}
{KhanUrmi}{176 {[144]}}

Example \vref{ex:154} neatly shows the difference in both function and in marking of the linker  \transc{ay} and the demonstrative \transc{ay}. 

An independent genitive pronoun can also occur as a \secn following the linker. This seems to further indicate that the genitive base \transc{did-} should not be confounded with the linker. Note also the optionality of the \cst* marking at the end of the \prim NP.

\acex{Noun Phrase}{Pronoun}{158}
{[ǰull-ət ⁺šultanul-a/ət] ay did-ew}
{clothes-\cst{} royalty-\free/\cst{} \lnk{} \gen-3\masc}
{his royal clothes}
{GarbellUrmi}{87}

\largerpage
\section{The double annexation construction (X-y.\poss\ \textsc{lnk} Y)} \label{ss:JUrm_double}

\citet[87]{GarbellUrmi} mentions that  \enquote{in rare cases} a double genitive construction can occur. In such cases, the \isi{apposition} between the \prim noun and the \lnk* is quite clear: 

\acex{Noun}{Noun}{160}
{tar-ew ay d-o gora}
{gate-\poss.3\masc{} \lnk{} \gen-\dem.\far.\masc{} man}
{that man's gate}
{GarbellUrmi}{87}





\section{Usage of the relativizer (X \textsc{rel} Y)} \label{ss:JUrm_Rel}

\subsection{Introduction}

The linker \transc{ay} is in \isi{complementary distribution} with  the \isi{relativizer} \transc{ki}, which appears only before clausal \secns.\footnote{This \isi{complementary distribution} is reminiscent of the alternation between \transc{ke} and \transc{ya} in \JSan discussed in \sref{ss:JSan_rel}. In the latter case, however, both forms serve as relativizers. According to \citet[171--172]{Garbell1965impact} the \isi{relativizer} is borrowed from \ili{Azeri} \ili{Turkish}, while the usage of the linker as well as the \cst* suffix stems from \isi{pattern replication} of Kurdish. \label{ft:JUrm_rel_influence}}

\acex{Noun}{Clause}{241}
{qúš ki\cb{} [baxtà \cb{}ila]}
{bird \rel\cb{} \hphantom{[}wife \cb{}\cop.3\fem}
{the bird who is the wife}
{KhanUrmi}{353 {[46]}}

\acex{Noun Phrase}{Clause}{165}
{[xa ⁺jahəl jwanqa], ki [atta ⁺matóy-le]}
{\indef{} young youth \rel{} now arrive.\inf-3\masc}
{a young man who has just reached maturity}
{GarbellUrmi}{88}

With the use of \transc{ki}, there is clearly no restriction as for the appearance of an explicit subject NP in the \isi{relative clause}, in contrast to clausal \secns following the \cst* marking, which may have such a restriction according to Garbell. Contrast the following example with \examples{166}{168}:

\acex{Noun}{Clause}{163}
{xa xabra ki [naš la ⁺miss-e ód-le]}
{\indef{} thing \rel{} man \neg{} can-3\masc{} \sbjv.do.\agent3\masc-\patient3\masc}
{a thing that no one can do}
{GarbellUrmi}{88}





The \prim may moreover be a pronoun:

\acex{Pronoun}{Clause}{243}
{ó ki\cb{} [la dhə́l-le g\cb{} qór-ət dad-éw]}
{3\sg{} \rel\cb{} \neg{} knocked-3\masc{} in\cb{} grave-\cst{} father-\poss.3\masc}
{the one who did not beat on the grave of his father}
{KhanUrmi}{356 {[69]}}

Interestingly, sometimes a clause-like complement lacking a finite verb can appear in this construction \citep[361]{KhanUrmi}.

\acex{Noun}{Infinitive phrase}{252}
{(⁺hudaé m\cb{}) [pə́lg-ət ⁺wə́rxa] kí [knəštá izalà]ˈ (der-í-wa gòlbara)}
{Jews from\cb{} half-\cst{} way \rel{} synagogue go.\inf{} return-3\pl-\pst{} back}
{(Jews turned back from halfway) along the road that they had gone to the synagogue}
{KhanUrmi}{361 {[157]}}

\acex{Pronoun}{Noun}{245}
{ána ki\cb{} [\parplus də́qna-xwára]} 
{1\sg{} \rel\cb{} \hphantom{[\parplus}beard-white}
{I, who am an elder}
{KhanUrmi}{356 {[17]}}

In the last two examples adding a \isi{copula} to the \secns would make them full clauses. In particular, the infinitive \foreign{izala}{go} could combine with a \isi{copula} to form the progressive tense. Thus, these cases may indicate that it is possible to omit the \isi{copula} in relative clauses.


\subsection{Adverbial \prims}
Some \isi{adverbial} \prims use the \isi{relativizer} construction to govern clausal \secns. This is the case with \foreign{magon}{like} in the following example (but this is optional; see \example{261}):

\largerpage
\acex{Adverbial}{Clause}{262}
{magon\cb{} kìˈ [k-yèt]ˈ}
{like\cb{} \rel{} \ind-know.2\masc}
{as you know}
{KhanUrmi}{373 {[51]}}\antipar



Of interest is the \isi{adverbial} \foreign{hal}{until} which uses the same construction both for clausal and nominal complements:
\newpage 

\acex{Adverbial}{Clause}{256}
{⁺hal\cb{} ki\cb{} [yá ⁺šultán-a -d\cb{}áy {Pahlawi}̀ˈ ədyè-le].ˈ}
{until\cb{} \rel\cb{} \dem.\near.\sg{} king-\cst{} -\cst\cb{}\lnk{} P. came-3\masc}
{until king (Reza Shah) Pahlavi came}
{KhanUrmi}{370 {[169]}}

\acex{Adverbial}{Noun Phrase}{257}
{⁺hal\cb{} kì [lel\cb{} xlulá]}
{until\cb{} \rel{} night.\cst{}\cb{} wedding}
{until the wedding night}
{KhanUrmi}{371 {[73]}}

The latter case may be interpreted as a clausal complement in which the  \isi{copula} has been omitted as in examples \ref{ex:252}--\ref{ex:245}. It seems more reasonable, however, to analyse the element \transc{ki} as if it had been integrated into the \isi{adverbial} \transc{hal}. The following example,  in which the whole expression \transc{hal ki} governs a \isi{genitive case}, corroborates this view: 

\acex{Adverbial}{Noun}{258}
{⁺hál ki\cb{} d-o\cb{} lelé (\cb{}š)}
{until \rel\cb{} \gen-\dem.\far.\masc\cb{} night \cb{}also}
{until that night}
{KhanUrmi}{371 {[82]}}

\largerpage
\subsection{Relativizer following the construct state (X-\textsc{cst} \textsc{rel} Y)}

The \rel*, like the \lnk*, may follow a \isi{head noun} which is marked by the \cst* suffix. In this case, as in \examples{166}{168}, it seems that the clause cannot have an NP subject argument.

\acex{Noun}{Clause}{113}
{naš-ət ki [lóka wé-lu]}
{people-\cst{} \rel{} there \cop.\pst-3\pl}
{the people who were there}
{GarbellUrmi}{55}

\acex{Noun}{Clause}{170}
{o brát-ət ki [midyá-wa-lu gall-ew]}
{\dem.\far.\sg{} girl-\cst{} \rel{} brought.\patient3\fem-\pst-\agent3\pl{} with-3\masc}
{that girl whom they had brought along with him}
{GarbellUrmi}{88}

In these cases, as those involving the \lnk* (see \ref{ss:JUrm_cst_lnk} \vpageref{ex:156bis}), the \prim noun may be analysed as exhibiting \concept{agreement in state} with the \isi{relativizer}, which is then understood to be syntactically in the \cst*.  Note, however, that the \cst* marking is not a simple variant of the construction,  as it restricts the class of \secn clauses following it to those that do not contain a subject NP. Thus, one may postulate that \transc{ki.\cst} is grammatically different from \transc{ki}, albeit the  forms are identical.

In some restricted cases, the \isi{relativizer} can follow the \cst* pronoun \transc{od} (see \sref{ss:JUrm_od}). \citet[374--5]{KhanUrmi} brings examples of this construction only with the prepositions \foreign{bod}{because} and \foreign{reš}{on}:

\acex{Pronoun}{Clause}{264}
{bo-d\cb{} ó-d kìˈ xălifá kotàk dahə́l-le}
{because-\cst\cb{} \dem-\cst{} \rel{} teacher blow beat.\agent3\masc-\patient3\masc}
{because of the fact that the teacher beats him}
{KhanUrmi}{374 {[139]}}

\acex{Pronoun}{Clause}{265}
{(mqé-lan) reš\cb{} d-ó-d kí  [⁺kalòˈ ki\cb{} bratá yadlà-waˈ jùwe šúla \cb{}š g-od-í-wa].ˈ}
{spoke-1\pl{} on\cb{} \gen-\dem-\cst{} \rel{}  bride \rel(when)\cb{} girl give\_birth.\fem-\pst{} different work \cb{}also \ind-do-3\pl-\pst}
{(We have spoken) about the fact that a daughter-in-law, when she gave birth to a girl, people acted differently.}
{KhanUrmi}{375 {[120]}}

As for \isi{adverbial} \prims, those which normally take the \cst* suffix will do so also when followed by a clausal complement introduced by the \isi{relativizer} \transc{ki}.

\acex{Adverbial}{Noun}{206}
{m\cb{}qulb-ə́t ki\cb{} [adé geb\cb{} marasxaná əl-lí xazè]}
{from\cb{}stead-\cst{} \rel\cb{} come.3\masc{} at\cb{} hospital \acc-1\sg{} see.3\masc}
{instead of coming to the hospital to see me}
{KhanUrmi}{569 {[148]}}


\subsection{Relativizer in construct state (X \textsc{rel-cst} Y)} \label{ss:JUrm_rel_cst}

Above, I claimed that the \isi{relativizer} \transc{ki} may sometimes be considered to be in \isi{construct state} syntactically speaking, although it is not marked as such morphologically. \JUrm seems to corroborate this claim by some cases in which the \isi{relativizer} is explicitly marked by a \cst* suffix: \transc{ki-t}. This seems to happen, though, only after a handful of  adverbials.  \citet{KhanUrmi} gives the following three cases:

\acex{Adverbial}{Clause}{235}
{⁺hál kí-t [idáyle léle -d\cb{} áy [pardìn šaroé]]}
{until \rel-\cst{} come.\inf.3\masc{} night -\cst\cb{} \lnk{} curtain untie.\inf}
{until the night of the releasing of the curtain came}
{KhanUrmi}{291 {[84]}}

\acex{Adverbial}{Clause}{289}
{⁺hal kí-t ya\cb{} bronàˈ yá axón-i zóra ləbb-éw zìlˈ}
{until \rel-\cst{} \dem\cb{} son \dem{} brother-1\sg{} small heart-\poss.3\masc{} went.3\masc}
{until the boy, my young brother, had fainted}
{KhanUrmi}{371 {[142]}}

\acex{Conjunction}{Clause}{254}
{bar ki-t ⁽⁺⁾dməx-lan lele}
{after \rel-\cst{} slept-1\pl{} night}
{after we went to bed}
{KhanUrmi}{369}

The restriction of this marker to these conjunctions only may indicate, however, that \transc{ki} is no longer felt as a \isi{relativizer} in these cases, but as a part of the conjunction. Indeed, there are other occurrences of the \isi{adverbial} \transc{hal-ki} (\examples{256}{258}), which suggest  the same view.




 \section{Juxtaposition (X Y.\opt{\agr})} \label{ss:JUrm_juxt}
 
\subsection{Introduction} 
 
 Juxtaposition of two nouns is used only marginally in \JUrm as a means of marking an \isi{attributive relation}:
 
 \acex{Noun}{Noun}{134}
 {naša tre reše}
 {man two heads}
 {the two-headed man}
 {GarbellUrmi}{86}
 
 \acex{Noun}{Noun}{215}
 {⁺səmha ilane}
 {festival trees}
 {Festival of the Trees (holiday of \textit{Tu bi-Shvat})}
 {KhanUrmi}{587}
 
 The first example may be motivated by the intervention of a \isi{numeral} between the \prim and the \secn (but contrast with \example{135}). The second example, on the other hand, is an idiom which refers to the Jewish holiday of \texthebrew{ט"ו בשבט} \textit{Tu bi-Shvat} (15\th\ of the month of Shvat). While the word \foreign{⁺səmha}{festivity} is borrowed from \ili{Hebrew} \citep[587]{KhanUrmi}, the expression as a whole is probably not borrowed from the \ili{Hebrew} parallel \texthebrew{שמחת אילנות} \transc{simḥa-t ʾilanot}, since in \ili{Hebrew} the word \transc{simḥa-t} is clearly marked by the \fem\ \cst* suffix \transc{-t}. 
 
 Juxtaposition as means of marking an AC is regularly found, on the other hand, with adjectival \secns. On the periphery of the AC system \isi{juxtaposition} is used with  nouns which are standing in \isi{apposition} with one another, as well as in \isi{adverbial} phrases. These cases are discussed in the following subsections.  
 

 \subsection{Adjectival \secns} \label{ss:JUrm_juxt_adj}
 
 Juxtaposition-cum-agreement is chiefly used in \JUrm for expressing \isi{adjectival attribution} (but often the adjective precedes the head, see  \sref{ss:JUrm_inverse}).
 
  While most adjectives of Aramaic origin agree with the \prim, some adjectives, mostly of foreign origin (mostly Kurdish or \ili{Azeri}), are invariable in form, and thus show a pure \isi{juxtaposition} pattern. Some other loan-adjectives do show agreement but only for number \citep[181]{KhanUrmi}.
 
 \acex{Noun}{Adjective}{114}
 {gi⁺lasta smuq-ta}
 {cherry(\fem) red-\fem}
 {the red cherry}
 {GarbellUrmi}{83}
 
 \acex{Noun}{Adjective}{292small}
 {tkana šušaband}
 {shop glass-covered(\invar)}
 {a glass-covered shop}
 {GarbellUrmi}{84}
 
 
 

 \subsection{Nominal quantification and apposition} \label{ss:JUrm_quant}
 
 Juxtaposition is regularly used when two nouns are in  \isi{apposition} with each other, but these cases do not fall normally under the definition of an AC used here. Yet some of these cases can be considered as borderline ACs, since one noun qualifies the other. This is for instance the case when the \isi{head noun} is part of a quantifying expression (Q. NP), as in the following examples \citep[see further][233--234]{KhanUrmi}. The following example illustrates the analytical ambiguity of such expressions: One the one hand, the \prim and the \secn are co-referential\footnote{To illustrate this, note that one may say \enquote{These are clothes} as well as \enquote{These are two sets}.} and share the same grammatical feature (plurality), and could thus be qualified as appositional to each other. On the other hand, the \secn \transl{clothes} clearly qualifies the \prim, as it designates the type of \transl{sets}.\footnote{The syntactic ambiguity of quantification is clearly manifested in \ili{Hebrew} morpho-syntax, where numerals appear in \cst* when followed by a definite nominal, and in \free* when followed by an indefinite one.}
 
 
 
 \acex{Q. Noun Phrase}{Noun}{123}
 {[tre daste] julle}
 {two sets clothes}
 {two sets of clothes}
 {GarbellUrmi}{85}
 
 In this class of examples one may also include the use of the \isi{numeral} classifier \foreign{danka}{unit} (\pl\ \transc{danke}). The usage of a classifier, as well as the classifier itself, is a matter-cum-\isi{pattern replication} from Kurdish (or possibly \ili{Azeri}), in which it originally means \transl{grain} \citep[172, \S 2.32.12.4(a)]{Garbell1965impact}.
 
 \acex
 {Q. Noun Phrase}{Noun}{2030}
 {[xa danka] baxta}
 {one unit woman}
 {one woman}
 {Garbell1965impact}{172}
 
 The \isi{apposition} between the two elements can be illustrated by the fact that the \prim can stand alone, without an explicit \secn:
 
 \acex
 {Q. Noun Phrase}{\zero}{2031}
 {isra danke}
 {ten unit.\pl}
 {ten (people or objects)}
 {Garbell1965impact}{172}
 
 Other quantification examples, however, are more clear-cut in that the \prim and the \secn are not co-referential, as in the following example:
 
 \acex{Q. Noun Phrase}{Noun}{124}
 {[kəmma \parplus bate] hudae}
 {few houses Jews}
 {a few Jewish houses}
 {GarbellUrmi}{85}
 
\largerpage
Another example of \isi{juxtaposition} verging on \isi{apposition} is the following one, where the \secn noun marks the biological sex of the \prim noun \foreign{quš}{bird}, which by itself has no inherent grammatical gender \citep[see glossary of][569]{KhanUrmi}. In this respect, the nominal \secns of the following example are not unlike adjectives (compare to example \vref{ex:241}):

\acex{Noun}{Noun}{218}
{qúš gorá (ba\cb{}) qúš baxtá (mar-è)}
{bird man \hphantom{(}to\cb{} bird woman say-3\masc}
{The male bird (says to) the female bird}
{KhanUrmi}{219 {[45]}}
 
\subsection{Adverbial \prims} \label{ss:JUrm_AdverbialHeads}

Adverbial \prims, which are not marked by the \cst* suffix, effectively yield a \isi{juxtaposition} construction. 

\acex{Preposition}{Noun}{183}
{b\cb{} šəmme}
{in\cb{} sky}
{in the sky}
{KhanUrmi}{192}

\acex{Preposition}{Pronoun}{267}
{bá\cb{} ma}
{for\cb{} what}
{Why?}
{KhanUrmi}{191}

\acex{Preposition}{Infinitive}{238}
{gal\cb{} rə̀qla,ˈ gal\cb{} zamòreˈ}
{with\cb{} dance.\inf{} with\cb{} sing.\inf}
{with dancing, with singing}
{KhanUrmi}{292 {[77]}} 

\acex{Conjunction}{Clause}{261}
{magón k-yé-tun}
{like \ind-know-2\pl}
{as you know}
{KhanUrmi}{372 {164}}

Compare these examples with the CSC of \examples{193}{266}. The last example can be contrasted as well with the ALC of \example{214}.

\section{Inverse juxtaposition (Y X)} \label{ss:JUrm_inverse}

The usage of inverse constructions, in which the \secn precedes the \prim, is not  uncommon in \JUrm, but it is restricted to two domains: adjectival and \isi{adverbial} attribution as well as complementation of verbal nouns.\footnote{Recall that in the example headings the categories of the constituents of the AC are always listed in the order \textbf{\Prim--\Secn}.}

\subsection{Adjectival and adverbial \secns}

Adjectives commonly precede the \isi{head noun} in \JUrm. This is attributed by \citet[172, \S 2.32.12 (2)]{Garbell1965impact} to \Azr influence.
Like post-nominal adjectives generally (see \sref{ss:JUrm_juxt_adj}), adjectives of Aramaic stock normally agree in gender and number with the \prim noun, while loan-adjectives are often uninfecting.

\acex{Noun}{Adjective}{115}
{xal-ta ⁺kalo}
{new-\fem{} bride}
{the new bride}
{GarbellUrmi}{83}

\acex{Noun}{Adjective}{290}
{kor naš-e}
{blind(\invar) people-\pl}
{blind people}
{Garbell1965impact}{167}

When two adjectives modify a noun, they are generally placed around the noun \citep[84]{GarbellUrmi}. It is the adjective with  larger scope which appears before the noun:

\acex{Noun Phrase}{Adjective}{292}
{zúr-ta [tkana šušaband]}
{small-\fem{} shop(\fem) glass-covered(\invar)}
{a small glass-covered shop}
{GarbellUrmi}{84}


Adverbials modifying an adjective appear before it. Consequently, an \isi{adverbial} may precede an adjective within an adjectival phrase preceding a \isi{head noun}:

\acex{Adjective}{Adverb}{119}
{[⁺raba xriwa] naš}
{much bad.\masc person}
{a very bad person}
{GarbellUrmi}{84}

\acex{Adjective}{\PP}{117}
{[[ba\cb{} taltoe] šbir-e] naš-e}
{for\cb{} hang.\inf{} good-\pl{} person-\pl}
{people good for hanging}
{GarbellUrmi}{84}

\acex{Adjective}{\PP}{116}
{[mən-nox biš zudda] naš-e}
{from-2\masc{} more brave(\invar) person-\pl}
{men braver than you}
{GarbellUrmi}{84}

One finds also the \isi{adverbial} \foreign{magon}{like} modifying directly a noun in this way:

\acex{Adverbial}{Noun}{142}
{magon-ox ⁺hasid-e}
{like-2\masc{} pious\_man-\pl}
{pious people like you}
{GarbellUrmi}{87} 

Finally, \isi{ordinals} may also precede a \prim noun. In contrast to the post-nominal placement of \isi{ordinals} (shown in \examples{130}{131}), the \prim noun is not marked as \cst* in this case. Note that the ordinal always has an invariable form.

\acex{Noun}{Ordinal}{133}
{tre-mənji gora}
{two-\ord{} man}
{The second man}
{KhanUrmi}{187}

\acex
{Noun Phrase}{Ordinal}{1927}
{tmanya-mənjì [lél-ət ay elá]ˈ}
{eight-\ord{} night-\cst{} \lnk{} festival}
{on the eight night of the festival}
{KhanUrmi}{217 {[104]}}

\subsection{Verbal nouns as \prims} \label{ss:JUrm_inverse_Verbal}

Verbal nouns, i.e.\ infinitives and \isi{participles} (active or resultative), have their complements preceding them, just as normal verbs do. In the \ili{Semitic} realm this is clearly an innovation. Indeed, the \JUrm OV order, available throughout the verbal system, is attributed by \citet[172, \S 2.32.22.1]{Garbell1965impact} to \Kur or \Azr influence.


\acex{Infinitive}{Noun}{232}
{⁺hatān masxoe}
{groom wash.\inf}
{the washing of the groom}
{KhanUrmi}{291}

\acex{Participle}{Noun}{137}
{masy-e doq-ana}
{fish-\pl{} catch-\ptcp}
{fish-catcher, fisherman}
{GarbellUrmi}{86}

\acex{Participle}{Adverb}{121}
{lóka hawy-an-e}
{here be-\ptcp-\pl}
{those present there}
{GarbellUrmi}{84}

\acex{Participle}{\PP}{143}
{reš suse ⁺rkiwa}
{on horse mounted.\resl}
{mounted upon a horse}
{GarbellUrmi}{87}\antipar 
\newpage 

Example \vref{ex:121} could be contrasted with example \vref{ex:122}, in which the adverb {follows} a \isi{participial} in \cst*. For a \isi{participial} phrase acting as the \prim of an AC see example \vref{ex:139}.

Of interest are also cases of definite direct objects of infinitives. These may be part of a \isi{prepositional phrase} headed by the accusative-marking \transc{əl}, and may also be indexed on the infinitive by a pronominal \isi{possessive suffix}: 

\acex{Infinitive}{Noun}{233}
{əl\cb{} d-o gora ⁺qatol-ew}
{\acc\cb{} \gen-\dem.\far.\masc{} man kill.\inf-\poss.3\masc}
{the killing of that man}
{KhanUrmi}{291}

Such cases accentuate the double nature of complements of infinitives, being both \gen* (as complements of nouns) and \acc* (as complements of verbs).

\section{Conclusions} \label{ss:JUrm_summary}

\JUrm presents an intricate and complex system of ACs, exploiting to a maximal extent the various marking possibilities. Indeed, there are examples with up to three simultaneous AC markers: a \prim marked by \cst*, a \secn marked by \gen* case, and in between a \lnk* (see \sref{ss:JUrm_lnk_gen}) 




The various AC markers of \JUrm and their possible combination are presented in \vref{tb:JUrm_loci}.

\begin{table}[h]
\centering
\begin{tabular}{lc c cr}
\toprule
  & 1					& 2		 			 & 3	&   \\		
  \midrule
X & ± \cst\ (\ed, Ap.)	& 
\begin{tabular}{c} 
± \lnk\ (\transc{ay}) \\  ± \rel.\opt{\cst}\ (\transc{ki}, \transc{ki-t}) \\ 
\end{tabular}

& ± \gen\ (\d, \transc{did-}) & Y \\
\bottomrule
\end{tabular}
\caption{AC markers  in \JUrm}\label{tb:JUrm_loci}
\end{table}

\largerpage
Where does this complexity stem from? A possible answer is that the language has borrowed through \isi{language contact} various AC marking strategies, which synchronically \textit{co-exist} in the same system. Indeed, some elements are clearly borrowed: The \transc{ki} \isi{relativizer} is borrowed both formally and functionally from \ili{Azeri} \ili{Turkish} \citep[172]{Garbell1965impact}. Moreover, as \citet[171--172]{Garbell1965impact} suggests, the \Kur \ez* construction may be the source of the \JUrm\ linker construction, relexified with native morphological material, and possibly also  the source of the suffixed \cst* marking. While these claims may be challenged (see \sref{ss:neo-CSC} and \sref{ss:genitive_development} for a discussion),  the result of the interaction of the  different processes involved, be they pattern or \isi{matter replication} and/or internal change,  
is an entangled and quite complex system. 

The most striking structural innovation in \JUrm is the co-occurrence of a \cst* \prim with a \lnk* in the ALC. This construction, unattested in previous strata of Aramaic (but found in  some other \ili{NENA} dialects in different forms; see \sref{ss:double}), presents an analytic challenge to the conceptual framework used here. I have attempted to resolve this difficulty  by postulating an \concept{agreement in state} rule or by re-analysing  \transc{ay} as a non-pronominal \lnk* (see \sref{ss:JUrm_cst_lnk}). It seems reasonable to assume that  \isi{language contact} must have played a certain role in the emergence of this not so typically \ili{Semitic} construction. 

The analytic difficulties revolving around the occurrences of the morpheme \transc{ay} (Is it a pronominal \lnk*? A \secn marker? Or simply a \dem*?) as well as the \phonemic{d} segment (Is it part of the \cst* suffix? A \gen* prefix? Both?), seem moreover to be typical of a system which is still in a state of flux. 

The use of the \isi{juxtaposition} construction for quantification,   involving the \isi{numeral} classifier \foreign{danka}{unit} (see \examples{2030}{2031}), must be a case of \concept{pattern-cum-matter replication} from Kurdish or \ili{Azeri}. For discussion of whether this construction is in general due to \isi{language contact} see \ref{ss:juxt_nom_quant}. 

A similar case where \isi{language contact} must be in play is the usage of the \isi{inverse \isi{juxtaposition} construction} (described in \sref{ss:JUrm_inverse}).  The positioning of adjectives before their nominal \prims is due to \ili{Azeri} influence, while the positioning of complements before their verbal nouns is related either to \Kur or to \Azr influence (or both).


Notwithstanding these changes, \JUrm has preserved some of the typical characteristics of a classical \ili{Semitic} system: First, it shows a clear demarcation between \isi{adjectival attribution} (expressed by \isi{juxtaposition-cum-agreement}) and nominal attribution (the CSC as well as the ALC). Second, the use of the CSC with clausal \secns, while absent in previous strata of Aramaic, is a  classical \ili{Semitic} pattern. Note, however, that it has been superseded to some extent by the use of the borrowed \isi{relativizer} \transc{ki}.\footnote{The usage of a dedicated \isi{relativizer} (differentiated from a more general \lnk*) is by itself not unprecedented in the \ili{Semitic} realm. For example, in \BHeb, one finds \transc{\texthebrew{שֶׁ} šɛ-} or \transc{\texthebrew{אֲשֶׁר} ʾăšɛr}  exclusively in the role of relativizers  \citep[331, \S 19.2]{WaltkeOconnor}.}

\largerpage
To conclude, compared to the dialects surveyed so far, \JUrm seems to present the most complex  system, rich in its variety of constructions, and the most innovative one compared to the \il{Aramaic!Classical}Classical Aramaic AC system. Yet it keeps also some conservative aspects typical of \ili{Semitic} languages. 







